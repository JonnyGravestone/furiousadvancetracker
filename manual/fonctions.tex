% Fichier de fonctions et de parametres pour LATEX
% Pour l'utiliser: % Fichier de fonctions et de parametres pour LATEX
% Pour l'utiliser: % Fichier de fonctions et de parametres pour LATEX
% Pour l'utiliser: % Fichier de fonctions et de parametres pour LATEX
% Pour l'utiliser: \include{fonctions}

% Prototypes des fonctions

\newcommand{\fatversion}{version 1.0.0 (rc1)}
\newcommand{\fatnextversion}{version 1.1.0}

% Definition des couleurs
\definecolor{vert}{rgb}{0.28,0.56,0.09} % #499118
\definecolor{bleu}{rgb}{0.26,0.45,0.72} % #4476ba
\definecolor{rouge}{rgb}{0.85,0.08,0.08} % #db1515
\newcommand{\FAT}{ \textcolor{vert}{F}\textcolor{bleu}{A}\textcolor{rouge}{T} }
\newcommand{\BURGERBIS}{ \textcolor{vert}{B}\textcolor{bleu}{u}\textcolor{rouge}{r}\textcolor{vert}{g}\textcolor{bleu}{e}\textcolor{rouge}{r} }
\newcommand{\BURGER}{ \textcolor{rouge}{Burger} }

% JAVA

\definecolor{code_string_JAVA}{rgb}{0.85,0.08,0.08} % #db1515
\definecolor{code_comments_JAVA}{rgb}{0.40,0.66,0.23} % #68aa3d
% XML

\definecolor{code_comment_XML}{rgb}{0.11,0.14,0.50} % #1f2480
\definecolor{code_tag_XML}{rgb}{0.30,0.60,0.58} % #4f9995
% Tableau
\definecolor{headertab}{rgb}{0.70, 0.70, 0.70}
\definecolor{impair}{rgb}{0.95, 0.95, 0.95}

\definecolor{bordureannotation}{rgb}{0.80, 0.86, 0.47}
\definecolor{fondannotation}{rgb}{0.98, 0.99, 0.83}

\newcommand{\POM}{\textcolor{code_key_XML}{pom.xml} }

% JAVA
% Usage: \JAVA
\newcommand{\JAVA}{
  \lstset{
    language=java,
    float=hbp,
    basicstyle=\small,
    identifierstyle=\color{black},
    keywordstyle=\bf \color{code_key_JAVA},
    stringstyle=\color{code_string_JAVA},
    commentstyle=\color{code_comments_JAVA},
    %declarationstyle=\bg,
    columns=flexible,
    tabsize=4,
    frame=single,
    rulesepcolor=\color[gray]{0.5},
    extendedchars=true,
    showspaces=false,
    showstringspaces=false,
    numbers=left,
    numberstyle=\tiny,
    breaklines=true,
    captionpos=b
  }
}
% XML
% Usage: \XML
\newcommand{\XML}{
  \lstset{
    language=xml,
    float=hbp,
    basicstyle=\small,
    identifierstyle=\color{black},
    keywordstyle=\bf \color{code_key_XML},
    stringstyle=\color{code_string_XML},
    commentstyle=\color{code_comment_XML},
    tagstyle=\color{code_tag_XML},
    columns=flexible,
    tabsize=2,
    frame=single,
    rulesepcolor=\color[gray]{0.5},
    extendedchars=true,
    showspaces=false,
    showstringspaces=false,
    numbers=left,
    numberstyle=\tiny,
    breaklines=true,
    captionpos=b,
    morekeywords={xmlns, xmlns:xsi, encoding}
  }
}
% SHELL
% Usage: \SHELL
\newcommand{\SHELL}{
  \lstset{
    language=csh,
    float=hbp,
    basicstyle=\small,
    identifierstyle=\color{black},
    keywordstyle=\bf \color{black},
    stringstyle=\color{red},
    commentstyle=\color{blue},
    columns=flexible,
    tabsize=2,
    frame=single,
    rulesepcolor=\color[gray]{0.5},
    extendedchars=true,
    showspaces=false,
    showstringspaces=false,
    numbers=left,
    numberstyle=\tiny,
    breaklines=true,
    captionpos=b,
    morekeywords={mvn,export}
  }
}


% -- Fonctions
% -- Insérer un backslash
% Usage: \bslash
\newcommand{\bslash}{\texttt{\symbol{92}}}

% -- Insérer un espace 
% Usage: \tbsp
\newcommand{\tbsp}{\rule{0pt}{18pt}}

% -- Insérer du code JAVA
% Usage: \IncludeJavaCode{nomfichier.java}{caption}
\newcommand{\IncludeJavaCode}[2]{
  \begin{center}
  \JAVA
    \begin{frame}
    
      \lstinputlisting[caption=#2]{#1}
      \end{frame}
  \end{center}
}
% -- Insérer du code XML
% Usage: \IncludeXMLCode{nomfichier.xml}{caption}
\newcommand{\IncludeXMLCode}[2]{
  \begin{center}
  \XML
    \begin{frame}
    
      \lstinputlisting[caption=#2]{#1}
      \end{frame}
  \end{center}
}
% -- Insérer une commande shell
% Usage: \IncludeShellCMD{nomfichier.sh}{caption}
\newcommand{\IncludeShellCMD}[2]{
  \begin{center}
  \SHELL
    \begin{frame}
    
      \lstinputlisting[caption=#2]{#1}
      \end{frame}
  \end{center}
}
% -- Insérer une annotation
% Usage: \Annotation{texte}
\newcommand{\Annotation}[1]{
  \begin{center}
    \fbox{
      \begin{minipage}{1.0\linewidth}
        \vspace{0.2cm}
        \sloppy{#1}
        \vspace{0.2cm}
      \end{minipage}
    }
  \end{center}
}

% -- Insérer une annotation dans une boite colorée
% Usage: \ColoredAnnotation{texte}
\newcommand{\ColoredAnnotation}[1]{
    \begin{center}
        %\fcolorbox{bordure}{couleur }{ . . . }
    	\fcolorbox{bordureannotation}{fondannotation}{
    	    \begin{minipage}{1.0\linewidth}
                \vspace{0.2cm}
                \sloppy{#1}
                \vspace{0.2cm}
            \end{minipage}    	
    	}
    \end{center}
}


% -- Insérer une image
% Usage : \Image{nomFichier}{scale}{caption}
\newcommand {\Image}[3]{
  \begin{figure}[H]
    \centering
    \includegraphics[scale=#2]{#1}
    \caption{#3}
  \end{figure}
}

% -- Insérer une image tournée
% Usage : \ImageRotated{nomfichier}{scale}{caption}{angle}
\newcommand {\ImageRotated}[4]{
  \begin{figure}[H]
    \centering
    \includegraphics[scale=#2,angle=#4]{#1}
    \caption{#3}
  \end{figure}
}

% -- Insérer une image simplement
% Usage: \SimpleImage{nomfichier}{scale}
\newcommand {\SimpleImage}[2]{
  \includegraphics[scale=#2]{#1}
}

% -- Ecrire du texte avec une image a droite
% Usage: \WriteTextWithImgOnRight{fichierimage}{scale}{texte}
\newcommand{\WriteTextWithImgOnRight}[3]{
  \parpic[r][b]{\SimpleImage{#1}{#2}} #3
}

% -- Ecrire du texte avec une image a gauche
% Usage: \WriteTextWithImgOnLeft{fichierimage}{scale}{texte}
\newcommand{\WriteTextWithImgOnLeft}[3]{
  \parpic[l][b]{\SimpleImage{#1}{#2}} #3
}


% Prototypes des fonctions

\newcommand{\fatversion}{version 1.0.0 (rc1)}
\newcommand{\fatnextversion}{version 1.1.0}

% Definition des couleurs
\definecolor{vert}{rgb}{0.28,0.56,0.09} % #499118
\definecolor{bleu}{rgb}{0.26,0.45,0.72} % #4476ba
\definecolor{rouge}{rgb}{0.85,0.08,0.08} % #db1515
\newcommand{\FAT}{ \textcolor{vert}{F}\textcolor{bleu}{A}\textcolor{rouge}{T} }
\newcommand{\BURGERBIS}{ \textcolor{vert}{B}\textcolor{bleu}{u}\textcolor{rouge}{r}\textcolor{vert}{g}\textcolor{bleu}{e}\textcolor{rouge}{r} }
\newcommand{\BURGER}{ \textcolor{rouge}{Burger} }

% JAVA

\definecolor{code_string_JAVA}{rgb}{0.85,0.08,0.08} % #db1515
\definecolor{code_comments_JAVA}{rgb}{0.40,0.66,0.23} % #68aa3d
% XML

\definecolor{code_comment_XML}{rgb}{0.11,0.14,0.50} % #1f2480
\definecolor{code_tag_XML}{rgb}{0.30,0.60,0.58} % #4f9995
% Tableau
\definecolor{headertab}{rgb}{0.70, 0.70, 0.70}
\definecolor{impair}{rgb}{0.95, 0.95, 0.95}

\definecolor{bordureannotation}{rgb}{0.80, 0.86, 0.47}
\definecolor{fondannotation}{rgb}{0.98, 0.99, 0.83}

\newcommand{\POM}{\textcolor{code_key_XML}{pom.xml} }

% JAVA
% Usage: \JAVA
\newcommand{\JAVA}{
  \lstset{
    language=java,
    float=hbp,
    basicstyle=\small,
    identifierstyle=\color{black},
    keywordstyle=\bf \color{code_key_JAVA},
    stringstyle=\color{code_string_JAVA},
    commentstyle=\color{code_comments_JAVA},
    %declarationstyle=\bg,
    columns=flexible,
    tabsize=4,
    frame=single,
    rulesepcolor=\color[gray]{0.5},
    extendedchars=true,
    showspaces=false,
    showstringspaces=false,
    numbers=left,
    numberstyle=\tiny,
    breaklines=true,
    captionpos=b
  }
}
% XML
% Usage: \XML
\newcommand{\XML}{
  \lstset{
    language=xml,
    float=hbp,
    basicstyle=\small,
    identifierstyle=\color{black},
    keywordstyle=\bf \color{code_key_XML},
    stringstyle=\color{code_string_XML},
    commentstyle=\color{code_comment_XML},
    tagstyle=\color{code_tag_XML},
    columns=flexible,
    tabsize=2,
    frame=single,
    rulesepcolor=\color[gray]{0.5},
    extendedchars=true,
    showspaces=false,
    showstringspaces=false,
    numbers=left,
    numberstyle=\tiny,
    breaklines=true,
    captionpos=b,
    morekeywords={xmlns, xmlns:xsi, encoding}
  }
}
% SHELL
% Usage: \SHELL
\newcommand{\SHELL}{
  \lstset{
    language=csh,
    float=hbp,
    basicstyle=\small,
    identifierstyle=\color{black},
    keywordstyle=\bf \color{black},
    stringstyle=\color{red},
    commentstyle=\color{blue},
    columns=flexible,
    tabsize=2,
    frame=single,
    rulesepcolor=\color[gray]{0.5},
    extendedchars=true,
    showspaces=false,
    showstringspaces=false,
    numbers=left,
    numberstyle=\tiny,
    breaklines=true,
    captionpos=b,
    morekeywords={mvn,export}
  }
}


% -- Fonctions
% -- Insérer un backslash
% Usage: \bslash
\newcommand{\bslash}{\texttt{\symbol{92}}}

% -- Insérer un espace 
% Usage: \tbsp
\newcommand{\tbsp}{\rule{0pt}{18pt}}

% -- Insérer du code JAVA
% Usage: \IncludeJavaCode{nomfichier.java}{caption}
\newcommand{\IncludeJavaCode}[2]{
  \begin{center}
  \JAVA
    \begin{frame}
    
      \lstinputlisting[caption=#2]{#1}
      \end{frame}
  \end{center}
}
% -- Insérer du code XML
% Usage: \IncludeXMLCode{nomfichier.xml}{caption}
\newcommand{\IncludeXMLCode}[2]{
  \begin{center}
  \XML
    \begin{frame}
    
      \lstinputlisting[caption=#2]{#1}
      \end{frame}
  \end{center}
}
% -- Insérer une commande shell
% Usage: \IncludeShellCMD{nomfichier.sh}{caption}
\newcommand{\IncludeShellCMD}[2]{
  \begin{center}
  \SHELL
    \begin{frame}
    
      \lstinputlisting[caption=#2]{#1}
      \end{frame}
  \end{center}
}
% -- Insérer une annotation
% Usage: \Annotation{texte}
\newcommand{\Annotation}[1]{
  \begin{center}
    \fbox{
      \begin{minipage}{1.0\linewidth}
        \vspace{0.2cm}
        \sloppy{#1}
        \vspace{0.2cm}
      \end{minipage}
    }
  \end{center}
}

% -- Insérer une annotation dans une boite colorée
% Usage: \ColoredAnnotation{texte}
\newcommand{\ColoredAnnotation}[1]{
    \begin{center}
        %\fcolorbox{bordure}{couleur }{ . . . }
    	\fcolorbox{bordureannotation}{fondannotation}{
    	    \begin{minipage}{1.0\linewidth}
                \vspace{0.2cm}
                \sloppy{#1}
                \vspace{0.2cm}
            \end{minipage}    	
    	}
    \end{center}
}


% -- Insérer une image
% Usage : \Image{nomFichier}{scale}{caption}
\newcommand {\Image}[3]{
  \begin{figure}[H]
    \centering
    \includegraphics[scale=#2]{#1}
    \caption{#3}
  \end{figure}
}

% -- Insérer une image tournée
% Usage : \ImageRotated{nomfichier}{scale}{caption}{angle}
\newcommand {\ImageRotated}[4]{
  \begin{figure}[H]
    \centering
    \includegraphics[scale=#2,angle=#4]{#1}
    \caption{#3}
  \end{figure}
}

% -- Insérer une image simplement
% Usage: \SimpleImage{nomfichier}{scale}
\newcommand {\SimpleImage}[2]{
  \includegraphics[scale=#2]{#1}
}

% -- Ecrire du texte avec une image a droite
% Usage: \WriteTextWithImgOnRight{fichierimage}{scale}{texte}
\newcommand{\WriteTextWithImgOnRight}[3]{
  \parpic[r][b]{\SimpleImage{#1}{#2}} #3
}

% -- Ecrire du texte avec une image a gauche
% Usage: \WriteTextWithImgOnLeft{fichierimage}{scale}{texte}
\newcommand{\WriteTextWithImgOnLeft}[3]{
  \parpic[l][b]{\SimpleImage{#1}{#2}} #3
}


% Prototypes des fonctions

\newcommand{\fatversion}{version 1.0.0 (rc1)}
\newcommand{\fatnextversion}{version 1.1.0}

% Definition des couleurs
\definecolor{vert}{rgb}{0.28,0.56,0.09} % #499118
\definecolor{bleu}{rgb}{0.26,0.45,0.72} % #4476ba
\definecolor{rouge}{rgb}{0.85,0.08,0.08} % #db1515
\newcommand{\FAT}{ \textcolor{vert}{F}\textcolor{bleu}{A}\textcolor{rouge}{T} }
\newcommand{\BURGERBIS}{ \textcolor{vert}{B}\textcolor{bleu}{u}\textcolor{rouge}{r}\textcolor{vert}{g}\textcolor{bleu}{e}\textcolor{rouge}{r} }
\newcommand{\BURGER}{ \textcolor{rouge}{Burger} }

% JAVA

\definecolor{code_string_JAVA}{rgb}{0.85,0.08,0.08} % #db1515
\definecolor{code_comments_JAVA}{rgb}{0.40,0.66,0.23} % #68aa3d
% XML

\definecolor{code_comment_XML}{rgb}{0.11,0.14,0.50} % #1f2480
\definecolor{code_tag_XML}{rgb}{0.30,0.60,0.58} % #4f9995
% Tableau
\definecolor{headertab}{rgb}{0.70, 0.70, 0.70}
\definecolor{impair}{rgb}{0.95, 0.95, 0.95}

\definecolor{bordureannotation}{rgb}{0.80, 0.86, 0.47}
\definecolor{fondannotation}{rgb}{0.98, 0.99, 0.83}

\newcommand{\POM}{\textcolor{code_key_XML}{pom.xml} }

% JAVA
% Usage: \JAVA
\newcommand{\JAVA}{
  \lstset{
    language=java,
    float=hbp,
    basicstyle=\small,
    identifierstyle=\color{black},
    keywordstyle=\bf \color{code_key_JAVA},
    stringstyle=\color{code_string_JAVA},
    commentstyle=\color{code_comments_JAVA},
    %declarationstyle=\bg,
    columns=flexible,
    tabsize=4,
    frame=single,
    rulesepcolor=\color[gray]{0.5},
    extendedchars=true,
    showspaces=false,
    showstringspaces=false,
    numbers=left,
    numberstyle=\tiny,
    breaklines=true,
    captionpos=b
  }
}
% XML
% Usage: \XML
\newcommand{\XML}{
  \lstset{
    language=xml,
    float=hbp,
    basicstyle=\small,
    identifierstyle=\color{black},
    keywordstyle=\bf \color{code_key_XML},
    stringstyle=\color{code_string_XML},
    commentstyle=\color{code_comment_XML},
    tagstyle=\color{code_tag_XML},
    columns=flexible,
    tabsize=2,
    frame=single,
    rulesepcolor=\color[gray]{0.5},
    extendedchars=true,
    showspaces=false,
    showstringspaces=false,
    numbers=left,
    numberstyle=\tiny,
    breaklines=true,
    captionpos=b,
    morekeywords={xmlns, xmlns:xsi, encoding}
  }
}
% SHELL
% Usage: \SHELL
\newcommand{\SHELL}{
  \lstset{
    language=csh,
    float=hbp,
    basicstyle=\small,
    identifierstyle=\color{black},
    keywordstyle=\bf \color{black},
    stringstyle=\color{red},
    commentstyle=\color{blue},
    columns=flexible,
    tabsize=2,
    frame=single,
    rulesepcolor=\color[gray]{0.5},
    extendedchars=true,
    showspaces=false,
    showstringspaces=false,
    numbers=left,
    numberstyle=\tiny,
    breaklines=true,
    captionpos=b,
    morekeywords={mvn,export}
  }
}


% -- Fonctions
% -- Insérer un backslash
% Usage: \bslash
\newcommand{\bslash}{\texttt{\symbol{92}}}

% -- Insérer un espace 
% Usage: \tbsp
\newcommand{\tbsp}{\rule{0pt}{18pt}}

% -- Insérer du code JAVA
% Usage: \IncludeJavaCode{nomfichier.java}{caption}
\newcommand{\IncludeJavaCode}[2]{
  \begin{center}
  \JAVA
    \begin{frame}
    
      \lstinputlisting[caption=#2]{#1}
      \end{frame}
  \end{center}
}
% -- Insérer du code XML
% Usage: \IncludeXMLCode{nomfichier.xml}{caption}
\newcommand{\IncludeXMLCode}[2]{
  \begin{center}
  \XML
    \begin{frame}
    
      \lstinputlisting[caption=#2]{#1}
      \end{frame}
  \end{center}
}
% -- Insérer une commande shell
% Usage: \IncludeShellCMD{nomfichier.sh}{caption}
\newcommand{\IncludeShellCMD}[2]{
  \begin{center}
  \SHELL
    \begin{frame}
    
      \lstinputlisting[caption=#2]{#1}
      \end{frame}
  \end{center}
}
% -- Insérer une annotation
% Usage: \Annotation{texte}
\newcommand{\Annotation}[1]{
  \begin{center}
    \fbox{
      \begin{minipage}{1.0\linewidth}
        \vspace{0.2cm}
        \sloppy{#1}
        \vspace{0.2cm}
      \end{minipage}
    }
  \end{center}
}

% -- Insérer une annotation dans une boite colorée
% Usage: \ColoredAnnotation{texte}
\newcommand{\ColoredAnnotation}[1]{
    \begin{center}
        %\fcolorbox{bordure}{couleur }{ . . . }
    	\fcolorbox{bordureannotation}{fondannotation}{
    	    \begin{minipage}{1.0\linewidth}
                \vspace{0.2cm}
                \sloppy{#1}
                \vspace{0.2cm}
            \end{minipage}    	
    	}
    \end{center}
}


% -- Insérer une image
% Usage : \Image{nomFichier}{scale}{caption}
\newcommand {\Image}[3]{
  \begin{figure}[H]
    \centering
    \includegraphics[scale=#2]{#1}
    \caption{#3}
  \end{figure}
}

% -- Insérer une image tournée
% Usage : \ImageRotated{nomfichier}{scale}{caption}{angle}
\newcommand {\ImageRotated}[4]{
  \begin{figure}[H]
    \centering
    \includegraphics[scale=#2,angle=#4]{#1}
    \caption{#3}
  \end{figure}
}

% -- Insérer une image simplement
% Usage: \SimpleImage{nomfichier}{scale}
\newcommand {\SimpleImage}[2]{
  \includegraphics[scale=#2]{#1}
}

% -- Ecrire du texte avec une image a droite
% Usage: \WriteTextWithImgOnRight{fichierimage}{scale}{texte}
\newcommand{\WriteTextWithImgOnRight}[3]{
  \parpic[r][b]{\SimpleImage{#1}{#2}} #3
}

% -- Ecrire du texte avec une image a gauche
% Usage: \WriteTextWithImgOnLeft{fichierimage}{scale}{texte}
\newcommand{\WriteTextWithImgOnLeft}[3]{
  \parpic[l][b]{\SimpleImage{#1}{#2}} #3
}


% Prototypes des fonctions

\newcommand{\fatversion}{version 1.3.0}
\newcommand{\fatnextversion}{version 1.4.0}

% Definition des couleurs
\definecolor{vert}{rgb}{0.28,0.56,0.09} % #499118
\definecolor{bleu}{rgb}{0.26,0.45,0.72} % #4476ba
\definecolor{rouge}{rgb}{0.85,0.08,0.08} % #db1515
\newcommand{\FAT}{ \textcolor{vert}{F}\textcolor{bleu}{A}\textcolor{rouge}{T} }
\newcommand{\BURGERBIS}{ \textcolor{vert}{B}\textcolor{bleu}{u}\textcolor{rouge}{r}\textcolor{vert}{g}\textcolor{bleu}{e}\textcolor{rouge}{r} }
\newcommand{\BURGER}{ \textcolor{rouge}{Burger} }

% JAVA

\definecolor{code_string_JAVA}{rgb}{0.85,0.08,0.08} % #db1515
\definecolor{code_comments_JAVA}{rgb}{0.40,0.66,0.23} % #68aa3d
% XML

\definecolor{code_comment_XML}{rgb}{0.11,0.14,0.50} % #1f2480
\definecolor{code_tag_XML}{rgb}{0.30,0.60,0.58} % #4f9995
% Tableau
\definecolor{headertab}{rgb}{0.70, 0.70, 0.70}
\definecolor{impair}{rgb}{0.95, 0.95, 0.95}

\definecolor{bordureannotation}{rgb}{0.80, 0.86, 0.47}
\definecolor{fondannotation}{rgb}{0.98, 0.99, 0.83}

\newcommand{\POM}{\textcolor{code_key_XML}{pom.xml} }

% JAVA
% Usage: \JAVA
\newcommand{\JAVA}{
  \lstset{
    language=java,
    float=hbp,
    basicstyle=\small,
    identifierstyle=\color{black},
    keywordstyle=\bf \color{code_key_JAVA},
    stringstyle=\color{code_string_JAVA},
    commentstyle=\color{code_comments_JAVA},
    %declarationstyle=\bg,
    columns=flexible,
    tabsize=4,
    frame=single,
    rulesepcolor=\color[gray]{0.5},
    extendedchars=true,
    showspaces=false,
    showstringspaces=false,
    numbers=left,
    numberstyle=\tiny,
    breaklines=true,
    captionpos=b
  }
}
% XML
% Usage: \XML
\newcommand{\XML}{
  \lstset{
    language=xml,
    float=hbp,
    basicstyle=\small,
    identifierstyle=\color{black},
    keywordstyle=\bf \color{code_key_XML},
    stringstyle=\color{code_string_XML},
    commentstyle=\color{code_comment_XML},
    tagstyle=\color{code_tag_XML},
    columns=flexible,
    tabsize=2,
    frame=single,
    rulesepcolor=\color[gray]{0.5},
    extendedchars=true,
    showspaces=false,
    showstringspaces=false,
    numbers=left,
    numberstyle=\tiny,
    breaklines=true,
    captionpos=b,
    morekeywords={xmlns, xmlns:xsi, encoding}
  }
}
% SHELL
% Usage: \SHELL
\newcommand{\SHELL}{
  \lstset{
    language=csh,
    float=hbp,
    basicstyle=\small,
    identifierstyle=\color{black},
    keywordstyle=\bf \color{black},
    stringstyle=\color{red},
    commentstyle=\color{blue},
    columns=flexible,
    tabsize=2,
    frame=single,
    rulesepcolor=\color[gray]{0.5},
    extendedchars=true,
    showspaces=false,
    showstringspaces=false,
    numbers=left,
    numberstyle=\tiny,
    breaklines=true,
    captionpos=b,
    morekeywords={mvn,export}
  }
}


% -- Fonctions
% -- Insérer un backslash
% Usage: \bslash
\newcommand{\bslash}{\texttt{\symbol{92}}}

% -- Insérer un espace
% Usage: \tbsp
\newcommand{\tbsp}{\rule{0pt}{18pt}}

% -- Insérer du code JAVA
% Usage: \IncludeJavaCode{nomfichier.java}{caption}
\newcommand{\IncludeJavaCode}[2]{
  \begin{center}
  \JAVA
    \begin{frame}

      \lstinputlisting[caption=#2]{#1}
      \end{frame}
  \end{center}
}
% -- Insérer du code XML
% Usage: \IncludeXMLCode{nomfichier.xml}{caption}
\newcommand{\IncludeXMLCode}[2]{
  \begin{center}
  \XML
    \begin{frame}

      \lstinputlisting[caption=#2]{#1}
      \end{frame}
  \end{center}
}
% -- Insérer une commande shell
% Usage: \IncludeShellCMD{nomfichier.sh}{caption}
\newcommand{\IncludeShellCMD}[2]{
  \begin{center}
  \SHELL
    \begin{frame}

      \lstinputlisting[caption=#2]{#1}
      \end{frame}
  \end{center}
}
% -- Insérer une annotation
% Usage: \Annotation{texte}
\newcommand{\Annotation}[1]{
  \begin{center}
    \fbox{
      \begin{minipage}{1.0\linewidth}
        \vspace{0.2cm}
        \sloppy{#1}
        \vspace{0.2cm}
      \end{minipage}
    }
  \end{center}
}

% -- Insérer une annotation dans une boite colorée
% Usage: \ColoredAnnotation{texte}
\newcommand{\ColoredAnnotation}[1]{
    \begin{center}
        %\fcolorbox{bordure}{couleur }{ . . . }
    	\fcolorbox{bordureannotation}{fondannotation}{
    	    \begin{minipage}{1.0\linewidth}
                \vspace{0.2cm}
                \sloppy{#1}
                \vspace{0.2cm}
            \end{minipage}
    	}
    \end{center}
}


% -- Insérer une image
% Usage : \Image{nomFichier}{scale}{caption}
\newcommand {\Image}[3]{
  \begin{figure}[H]
    \centering
    \includegraphics[scale=#2]{#1}
    \caption{#3}
  \end{figure}
}

% -- Insérer une image tournée
% Usage : \ImageRotated{nomfichier}{scale}{caption}{angle}
\newcommand {\ImageRotated}[4]{
  \begin{figure}[H]
    \centering
    \includegraphics[scale=#2,angle=#4]{#1}
    \caption{#3}
  \end{figure}
}

% -- Insérer une image simplement
% Usage: \SimpleImage{nomfichier}{scale}
\newcommand {\SimpleImage}[2]{
  \includegraphics[scale=#2]{#1}
}

% -- Ecrire du texte avec une image a droite
% Usage: \WriteTextWithImgOnRight{fichierimage}{scale}{texte}
\newcommand{\WriteTextWithImgOnRight}[3]{
  \parpic[r][b]{\SimpleImage{#1}{#2}} #3
}

% -- Ecrire du texte avec une image a gauche
% Usage: \WriteTextWithImgOnLeft{fichierimage}{scale}{texte}
\newcommand{\WriteTextWithImgOnLeft}[3]{
  \parpic[l][b]{\SimpleImage{#1}{#2}} #3
}
