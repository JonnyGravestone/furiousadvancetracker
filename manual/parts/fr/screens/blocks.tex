Dans cet écran, vous pouvez éditer le contenu d'une séquence : chaque séquence peut contenir de 0 à 16 blocks.
Lorsque \FAT jouera la séquence, le logiciel commencera par lire le premier block pour descendre jusqu'au dernier emplacement vide.
Si \FAT rencontre un emplacement vide lors de la lecture alors il remonte au premier emplacement.
Si aucun block n'est présent dans la séquence en cours, alors \FAT ne la jouera pas (elle sera sautée).
\medskip

Pour faire simple, considérez cet écran comme une liste de blocks contenue dans une séquence (dont le numéro est affiché à droite de l'écran).

\Image{images/screen_blocks_complete}{1.0}{L'écran BLOCKS}

\subsubsection{Commandes}
% Configuration du tableau
\tablehead{\hline \rowcolor{headertab} {\bf Touche(s)} & {\bf Effet} \\ \hline}
\tabletail{\hline \multicolumn{2}{|l|} {\small ...} \\ \hline}
\tablelasttail{\hline}
\begin{supertabular}{|l|p{11cm}|}
\hline
    {\bf SELECT} & Passer en mode popup \\
    \hline
    {\bf R+L} & Afficher l'écran d'aide \\
    \hline
    {\bf START} & Lancer/stopper la lecture des blocks dans la séquence en cours d'édition \\
    \hline
    {\bf L+HAUT} & Déplacer le curseur tout en haut \\
    \hline
    {\bf L+BAS} & Déplacer le curseur tout en bas \\
    \hline
    {\bf A} & Inscrire le dernier numéro de block dans la case désignée \\
    \hline
    {\bf A+DIRECTION} & Modifier le numéro de block dans la case désignée \\
    \hline
    {\bf L+A} & Trouver un nouveau numéro de block disponible et l'inscrire \\
    \hline
    %{\bf R+A} & Non utilisé pour le moment \\
    %\hline
    {\bf B} & SI CASE NON VIDE | Couper le block \\
    \hline
    {\bf B} & SI CASE VIDE | Coller le block précédemment coupé \\
    \hline
    {\bf L+B} & SI CASE NON VIDE | Changer le numéro de block en copiant le contenu \\
    \hline
    {\bf L+B} & SI CASE VIDE | Coller le block précédemment coupé en changeant son numéro \\
    \hline
    {\bf R+DROITE/GAUCHE} & Changer le numéro de séquence en cours d'édition \\
    %\hline
    % {\bf R+HAUT/BAS} & Sauter à la séquence précédente/suivante dans le séquenceur \\
\hline
\end{supertabular}
