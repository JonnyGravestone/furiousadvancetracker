Cet écran se subdivise en 5: il existe autant de types d'instrument dans \FAT.
\medskip

\begin{itemize}
    \item{le type PULSE}
    \item{le type WAVE}
    \item{le type NOISE}
    \item{le type KIT/SAMPLE}
    \item{le type OSCILLATOR}
\end{itemize}\medskip

 Chaque instrument est spécialisé dans la lecture de notes {\bf sur un certain canal}.
 Rappelez-vous, l'écran SONG est composé de 6 colonnes.
 Ces 6 colonnes représentent en fait chacun des 6 canaux de la GBA. Ainsi,
 \medskip

 \begin{itemize}
    \item{les instruments PULSE sont spécialistes du canal 1 et 2: le PU1 et PU2}
    \item{les instruments WAVE sont utiles pour le canal 3 uniquement : le canal WAV}
    \item{les instrument NOISE sont, eux, prévus pour le canal 4: le canal NOI}
    \item{les instruments de type KIT/SAMPLE officient sur le canal 5 et 6: le SNA et SNB}
    \item{enfin, les OSCILLATOR génèrent logiciellement un son basique (sinusoidal, carré, triangulaire ou en dent de scie)}
 \end{itemize}\medskip

 Mais \FAT est suffisamment souple pour permettre l'utilisation d'un instrument dans un canal qui ne lui est pas dédié:
 le comportement peut cependant être surprenant ... ou inaudible.
 A vous de voir ce que vous souhaitez produire comme sonorités.

\subsubsection{Commandes}

% Configuration du tableau
\tablehead{\hline \rowcolor{headertab} {\bf Touche(s)} & {\bf Effet} \\ \hline}
\tabletail{\hline \multicolumn{2}{|l|} {\small ...} \\ \hline}
\tablelasttail{\hline}
\begin{supertabular}{|l|p{11cm}|}
\hline
    {\bf SELECT} & Passer en mode popup \\
    \hline
    {\bf R+L} & Afficher l'écran d'aide \\
    \hline
    {\bf START} & Lancer/stopper la lecture des notes du block actuellement en cours d'édition \\
    \hline
    {\bf B} & Jouer la note dans le simulateur \\
    \hline
    {\bf B+DIRECTION} & Modifier la valeur du simulateur \\
    \hline
    {\bf A+DIRECTION} & Modifier la valeur sous le curseur \\
    \hline
    {\bf R+DROITE/GAUCHE} & Changer le numéro d'instrument en cours d'édition \\
    \hline
    {\bf L+DROITE/GAUCHE} & Changer le type d'instrument \\
\hline
\end{supertabular}

\subsubsection{Instrument de type PULSE}

\Image{images/screen_instrument_pulse}{1.0}{Type PULSE}

\subsubsection{Paramètres}

\paragraph{Enveloppe: "Volume"} de 0 à F.
Cette valeur permet de régler le volume des notes attachées à cet instrument.
\medskip

\begin{itemize}
    \item{0: muet}
    \item{F: à fond !}
\end{itemize}

\paragraph{Enveloppe: "Direction"} \SimpleImage{images/envelope_direction_down}{1.0} ou \SimpleImage{images/envelope_direction_up}{1.0}.
Indique si le son monte ou descend.

\paragraph{Enveloppe : "Steptime"} de 0 à 7.
Agit sur la longueur du pas de l'onde (en gros).
\medskip

\begin{itemize}
    \item{0: paramètre non pris en compte}
    \item{1: assez court}
    \item{7: très court}
\end{itemize}

\paragraph{Enveloppe : "Wave"} \SimpleImage{images/envelope_waveduty_0}{1.0} ou \SimpleImage{images/envelope_waveduty_1}{1.0} ou \SimpleImage{images/envelope_waveduty_2}{1.0} ou \SimpleImage{images/envelope_waveduty_3}{1.0}.
Modifie la forme de l'onde jouée par la GameboyAdvance.

\paragraph{Paramètre : "Timed"} 0 ou 1.
\medskip

\begin{itemize}
    \item{0: le son n'est pas temporisé}
    \item{1: le son est temporisé, réglez la durée avec le paramètre "Soundlength"}
\end{itemize}

\paragraph{Paramètre : "Soundlength"} de 0 à 3F.
Ce paramètre n'est accessible que si "Timed" à été positionné sur 1.
\medskip

\begin{itemize}
    \item{0: le son est "infini"}
    \item{3f: le son est très (très... voire trop) court}
\end{itemize}

\paragraph{Paramètre : "Output"} Choisissez le mode de sortie du son.
\medskip

\begin{itemize}
    \item{L: le son sort à gauche (LEFT)}
    \item{R: le son sort à droite (RIGHT)}
    \item{RL: le son sort à droite et à gauche (LEFT/RIGHT)}
    \item{VIDE: le son ne sort plus (muet)}
\end{itemize}

\paragraph{Paramètre : "Sweep"} de 0 à 7F.
Ajoute un effet particulier sur la note (le mieux est encore de tester).
N'oubliez pas que ce paramètre ne s'applique sur sur le channel PU1.

\paragraph{Simulator} Changez la note inscrite dans cet espace:
        celle-ci sera jouée avec les paramètres de l'instrument que vous êtes en train d'éditer.

\subsubsection{Instrument de type WAVE}

\Image{images/screen_instrument_wave}{0.5}{Type WAVE}

\subsubsection{Paramètres}

\paragraph{Enveloppe : Volume} de 0 à 4.
Cette valeur permet de régler le volume des notes attachées à cet instrument.

\paragraph{Paramètre : "Timed"} 0 ou 1.
\medskip

\begin{itemize}
        \item{0: le son n'est pas temporisé}
        \item{1: le son est temporisé, réglez la durée avec le paramètre "Soundlength"}
    \end{itemize}

\paragraph{Paramètre : "Length"} de 0 à FF.
Ce paramètre n'est accessible que si "Timed" a été positionné sur 1.
\medskip

\begin{itemize}
        \item{00: le son est "infini"}
        \item{eb: le son commence à être court}
        \item{ff: le son est très (très très) court}
    \end{itemize}

\paragraph{Paramètre : "Voice"} de 0 à 17.
\FAT inclut 23 types d'instruments WAV (en héxadécimal, 17h = 23).
Depuis la version 1.2.0, il est possible de customiser et d'utiliser ses propres voix.

\paragraph{Paramètre : "Bank"} 0 ou 1.
Le canal WAV de la GBA est capable de charger 2 banks dans sa mémoire.
Vous pouvez décider quelle bank sera jouée,
sachant que la bank est directement chargée depuis le paramètre "Voice" juste au dessus (chaque "Voice" possède 2 banks).

\paragraph{Paramètre : "Bankmode"} "SIN" ou "DUA".
Le canal WAV est capable de faire une synthèse des 2 banks chargées en mémoire (le mode DUA = Dual) ou bien de n'en jouer qu'une seule (le mode SIN = Single).
Notez que si vous réglez ce paramètre avec SIN, \FAT jouera la bank sélectionnée avec le paramètre "Bank".

\paragraph{Paramètre : "Output"} Choisissez le mode de sortie du son.
\medskip

\begin{itemize}
    \item{L: le son sort à gauche (LEFT)}
    \item{R: le son sort à droite (RIGHT)}
    \item{RL: le son sort à droite et à gauche (LEFT/RIGHT)}
    \item{VIDE: le son ne sort plus (muet)}
\end{itemize}

\paragraph{Custom wave} \FAT vous permet de stocker et d'utiliser 3 voix customisées.
Pour plus d'informations, référez vous à la section \hyperref[subsec:customwave]{"L'écran CUSTOM WAVE"}.
\medskip

\begin{itemize}
  \item{un 'X' signifie que l'instrument n'utilise pas de voix customisée. La voix définit par \FAT est donc utilisée.}
  \item{changez ce 'X' pour une valeur comprise entre 0 et 2 pour utiliser la voix correspondante}
  \item{cette voix sélectionnée peut être éditée en appuyant sur "Go" (juste en dessous). \FAT vous affiche alors l'écran d'édition pour la voix sélectionnée.}
  \item{si une voix n'a jamais été éditée auparavant, elle est initialisée avec les données de la voix courante de l'instrument.}
\end{itemize}\medskip

\tablehead{\hline \rowcolor{headertab} {\bf Touche(s)} & {\bf Effet} \\ \hline}
\tabletail{\hline \multicolumn{2}{|l|} {\small ...} \\ \hline}
\tablelasttail{\hline}
\begin{supertabular}{|l|p{11cm}|}
    \hline
    {\bf A+DROITE/GAUCHE} & Changer le numéro de voix custom sélectionnée \\
    \hline
    {\bf A+HAUT/BAS} & Annuler la customisation sur l'instrument (reset à 'X') \\
\hline
\end{supertabular}
\medskip

\paragraph{Simulator} Changez la note inscrite dans cet espace:
    celle-ci sera jouée avec les paramètres de l'instrument que vous êtes en train d'éditer.

\subsubsection{Instrument de type NOISE}

\Image{images/screen_instrument_noise}{1.0}{Type NOISE}

\subsubsection{Paramètres}

\paragraph{Enveloppe : "Volume"} de 0 à F.
Cette valeur permet de régler le volume des notes attachées à cet instrument.
\medskip

\begin{itemize}
    \item{0: muet}
    \item{F: à fond !}
\end{itemize}

\paragraph{Enveloppe : "Direction"} \SimpleImage{images/envelope_direction_down}{1.0} ou \SimpleImage{images/envelope_direction_up}{1.0}.
    Indique si le son monte ou descend.

\paragraph{Enveloppe : "Steptime"} de 0 à 7.
Agit sur la longueur du pas de l'onde (en gros).

\paragraph{Enveloppe : "Polystep"} 0 ou 1.
Modifie la facon dont le générateur de bruit fonctionne.
\medskip

\begin{itemize}
    \item{0: bruit simple}
    \item{1: bruit qui grince}
\end{itemize}

\paragraph{Paramètre : "Timed"} 0 ou 1.
\medskip

\begin{itemize}
        \item{0: le son n'est pas temporisé}
        \item{1: le son est temporisé, réglez la durée avec le paramètre "Length"}
    \end{itemize}

\paragraph{Paramètre : "Length"} de 0 à 3F.
Ce paramètre n'est accessible que si "Timed" à été positionné sur 1.
\medskip

\begin{itemize}
    \item{0: le son est "infini"}
    \item{3f: le son est très (très... voire trop) court}
\end{itemize}

\paragraph{Paramètre : "Output"} Choisissez le mode de sortie du son.
\medskip
\begin{itemize}
    \item{L: le son sort à gauche (LEFT)}
    \item{R: le son sort à droite (RIGHT)}
    \item{RL: le son sort à droite et à gauche (LEFT/RIGHT)}
    \item{VIDE: le son ne sort plus (muet)}
\end{itemize}

\paragraph{Simulator} Changez la note inscrite dans cet espace:
            celle-ci sera jouée avec les paramètres de l'instrument que vous êtes en train d'éditer.

\subsubsection{Instrument de type SAMPLE}

\FAT est capable de lire des samples sur les canaux SNA et SNB.
Un sample est un morceau sonore de courte durée inclus dans une collection.
Une série de collection est incluse par défaut dans \FAT :\medskip

\begin{itemize}
    \item {DEV-MESS : quelques samples de test pour le développement}
    \item {LS-MGDRV : des samples Megadrive fournis par Nullsleep (\href{http://little-scale.blogspot.fr/2008/08/sega-mega-drive-sample-pack.html}{Site de Nullsleep})}
    \item {C64DRUMS : drums et snares en provenance du Commodore64}
    \item {YM-DRUMS : drums et snares en provenance de l'Atari ST}
    \item {BATTERY : des kicks, drums et snares d'une batterie "classique"}
    \item {RHYTHMS : quelques ryhtmes complets (mais encore non exploitable car mal samplé)}
    \item {TR-606-7 : sample de cette machine mythique}
\end{itemize} \medskip

Il est assez facile de rajouter votre propre collection de samples. Pour plus d'informations, consultez la section \hyperref[sec:addsamples]{"Manipulation de la ROM pour l'ajout de vos collections de samples"}.

\Image{images/screen_instrument_sample}{0.5}{Type SAMPLE/KIT}

\subsubsection{Paramètres}

\paragraph{Kit collection : "Name"} Le nom de la collection de sample dans laquelle piocher.

\paragraph{Enveloppe : "Volume"} Volume affecté au sample : 50\% (0) ou 100\% (1).

\paragraph{Paramètre : "Loop"} Le sample est t-il joué en boucle ?
                                Si "YES", alors la lecture du sample est répétée.

\paragraph{Paramètre : "Timed"} Définit si le sample sera coupé ou non.

\paragraph{Paramètre : "Length"} Si "Timed" est à 1, cette valeur définit la coupure en pourcentage de la durée totale du sample.

\paragraph{Paramètre : "Offset"} Offset de départ pour la lecture du sample.
                                En pourcentage de la durée totale du sample.

\paragraph{Paramètre : "Output"} Choisissez le mode de sortie du son.
\medskip

\begin{itemize}
    \item{L: le son sort à gauche (LEFT)}
    \item{R: le son sort à droite (RIGHT)}
    \item{RL: le son sort à droite et à gauche (LEFT/RIGHT)}
    \item{VIDE: le son ne sort plus (muet)}
\end{itemize}

\paragraph{Simulator} Changez la note inscrite dans cet espace:
        celle-ci sera jouée avec les paramètres de l'instrument que vous êtes en train d'éditer.

\subsubsection{Instrument de type OSCILLATOR}

\FAT est capable de générer le son simple d'un oscillateur. Ces instruments seront surement utilisés un jour
pour la synthèse FM. En attendant, ils restent utilisables en tant que tel dans les canaux SNA et SNB.

\ColoredAnnotation {Vous remarquerez assez rapidement que les oscillateurs de type sinusoidal n'ont pas un son "parfait".
Cela est dû à 2 choses. La première étant que le processeur de la GBA ne gère pas le calcul flottant (nombres avec virgules).
Le calcul d'une onde sinusoidale demande une certaine précision que l'on ne peut donc pas obtenir nativement et facilement.
La seconde chose : il reste néanmoins possible de faire du calcul flottant, mais de manière logicielle. Ce genre de calcul n'est cependant pas
encore utilisé dans \FAT. Il y a donc encore une marge de progression possible pour ce sujet.}

\Image{images/screen_instrument_oscillator}{0.5}{Type OSCILLATOR}

\subsubsection{Paramètres}

\paragraph{Oscillator : "OSC Shape"} La forme d'onde à générer. \FAT gère 4 types d'ondes :\medskip

\begin{itemize}
    \item{\SimpleImage{images/osc_shape_sinus}{1.0} sinus}
    \item{\SimpleImage{images/osc_shape_square}{1.0} carré}
    \item{\SimpleImage{images/osc_shape_triangle}{1.0} triangle}
    \item{\SimpleImage{images/osc_shape_sawtooth}{1.0} dents de scie}
\end{itemize}\medskip

\paragraph{Oscillator : "Amplitude"} Ce paramètre n'est pas encore disponible à l'utilisation.

\paragraph{Enveloppe : "Volume"} Volume affecté au sample : 50\% (0) ou 100\% (1).

\paragraph{Paramètre : "Timed"} Définit si le son sera coupé ou non.

\paragraph{Paramètre : "Length"} Si "Timed" est à 1, cette valeur définit la coupure en pourcentage de la durée totale du son.

\paragraph{Paramètre : "Output"} Choisissez le mode de sortie du son.
\medskip

\begin{itemize}
    \item{L: le son sort à gauche (LEFT)}
    \item{R: le son sort à droite (RIGHT)}
    \item{RL: le son sort à droite et à gauche (LEFT/RIGHT)}
    \item{VIDE: le son ne sort plus (muet)}
\end{itemize}
