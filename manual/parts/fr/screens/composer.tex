Imaginez cet écran comme un mini-piano ou un sampler: le COMPOSER va transformer votre Gameboy Advance en machine a improviser des sons !

\Image{images/screen_composer}{1.0}{L'écran COMPOSER}

Le principe est simple.
L'écran vous présente 8 emplacements de notes.
Chacune de ses notes est affectée à un bouton (seuls les boutons SELECT et START ne sont pas utilisables).
Si vous regardez en bas de l'écran, le mot "UNLOCKED" est écrit:
    il signifie que vous êtes en mode "édition" et que vous pouvez écrire des notes comme vous le feriez dans un BLOCK (avec A).
\medskip

Une fois que vous avez écrit toutes vos notes (qui peuvent chacune avoir un instrument différent), appuyez sur START.
Le mode passe à "LOCKED". Vous êtes maintenant en mode "joueur".
\medskip

Appuyez sur une touche: la note assignée est jouée en temps réel.
Changez de touche pour jouer une autre note. Simple n'est ce pas ?
\medskip

Notez aussi que ces données seront enregistrées dans chacune de vos chansons.
Vous pourrez préparer votre set de note à la maison et improviser en live sans forcer !
\medskip

Essayez ! composez quelques séquences puis essayez de jouer avec le COMPOSER en même temps !
\medskip

\subsubsection{Paramétrage}

Le composer peut également être paramétré:

\Image{images/parameters_composer}{1.0}{Les paramètres de l'écran COMPOSER}

\paragraph{Transpose} de 0 à FF.
Ajoute une valeur de transpose.
Cette valeur s'ajoutera à celle configurée pour le projet dans sa globalité.
Notez que toutes les notes du composer seront transposées avec cette valeur.

\paragraph{Key repeat} de 0 à FF.
Règle le temps d'attente entre chaque lecture d'une note du composer.
Plus cette valeur est élevée, plus le temps d'attente entre l'appui de deux touches sera grand.
Laissez cette valeur à zéro pour désactiver le temps d'attente.

\subsubsection{Commandes}

% Configuration du tableau
\tablehead{\hline \rowcolor{headertab} {\bf Touche(s)} & {\bf Effet} \\ \hline}
\tabletail{\hline \multicolumn{2}{|l|} {\small ...} \\ \hline}
\tablelasttail{\hline}
\begin{supertabular}{|l|p{11cm}|}
\hline
    {\bf SELECT} & Passer en mode popup \\
    \hline
    {\bf R+L} & Afficher l'écran d'aide \\
    \hline
    {\bf START} & Changer le mode (UNLOCKED/LOCKED) \\
    \hline
    {\bf TOUCHES} & LOCKED | Jouer la note correspondante \\
    \hline
    {\bf A+DIRECTION} & UNLOCKED | Changer la valeur de la note/instrument \\
\hline
\end{supertabular}
