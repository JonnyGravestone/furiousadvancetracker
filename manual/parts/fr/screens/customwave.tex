\label{subsec:customwave}

L'objectif de cet écran est de permettre l'édition de courbe pour les instruments de type WAV.

\ColoredAnnotation{Cet écran est une première version. L'affichage de la courbe est présentée de façon verticale pour des raisons d'affichage ce qui n'est pas
forcément des plus pratique. L'affichage sera basculé en horizontal dans les prochaines versions de \FAT (il faut juste que je trouve comment organiser tout ça).}

\Image{images/screen_custom_wave_editor}{0.5}{L'écran CUSTOM WAVE}

\subsubsection{Commandes}
% Configuration du tableau
\tablehead{\hline \rowcolor{headertab} {\bf Touche(s)} & {\bf Effet} \\ \hline}
\tabletail{\hline \multicolumn{2}{|l|} {\small ...} \\ \hline}
\tablelasttail{\hline}
\begin{supertabular}{|l|p{11cm}|}
    \hline
    {\bf SELECT} & Passer en mode popup \\
    \hline
    {\bf A+DIRECTION} & Modifier la valeur sous le curseur \\
    \hline
    {\bf R+A+DIRECTION} & Changer toutes les valeurs simultanément \\
    \hline
    {\bf START} & Lancer/stopper la lecture depuis le séquenceur \\
    \hline
    {\bf B} & Jouer la note dans le simulateur \\
    \hline
    {\bf B+DIRECTION} & Modifier la valeur du simulateur \\
\end{supertabular}
