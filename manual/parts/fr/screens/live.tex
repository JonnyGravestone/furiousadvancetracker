Ici, vous allez pouvoir jouer votre chanson comme vous le voulez.
C'est ce mode qui est à priviligier si vous voulez mouiller le maillot sur scène.
Si vous préférez simplement appuyer sur START dans l'écran SONG et danser avec le public, libre à vous !
Sachez cependant que le mode LIVE est là pour vous permettre de modifier la mécanique de vos morceaux.

\Image{images/screen_live}{1.0}{L'écran LIVE se présente comme l'écran SONG}

Le séquenceur est à nouveau affiché mais cette fois il n'est pas question d'éditer les séquences sur le tableau.
Par contre vous allez pouvoir démarrer chacun des channels dans l'ordre qui vous chante.
Une fois un channel démarré, il est possible d'effectuer plusieurs actions :
\medskip

\begin{itemize}
    \item{modifier le volume du channel}
    \item{changer la transposition sur le channel}
    \item{stopper sa lecture}
\end{itemize}
\medskip

Notez que l'écran LIVE propose deux modes : le mode MANUAL (MAN) et le mode AUTO (AUT).
En mode MANUAL, FAT n'avance PAS dans le séquenceur. La même séquence sur le channel est rejouée à l'infini: c'est à vous de faire progresser le morceau.
A l'inverse, en mode AUTO, FAT se comporte comme dans l'écran SONG : tous les channels en cours de lecture progressent normalement.
\newpage % hack pourri
\subsubsection{Commandes}
\begin{supertabular}{|l|p{10cm}|}
    \hline
    {\bf SELECT} & Passer en mode popup \\
    \hline
    {\bf R+L} & Afficher l'écran d'aide \\
    \hline
    {\bf START} & Lance/arrête la lecture du channel sous le curseur \\
    \hline
    {\bf R+START} & Lance/arrête la lecture de tous les channels \\
    \hline
    {\bf R+BAS} & Passer dans la partie configuration de l'écran (modification du volume, tsp, tempo et mode) \\
    \hline
    {\bf R+HAUT} & Passer dans la partie séquenceur de l'écran \\
    \hline
    {\bf A+DIRECTION} & DANS LA PARTIE CONFIGURATION - Modifie la valeur du paramètre sous le curseur \\
    \hline
    {\bf A+L+DIRECTION} & DANS LA PARTIE CONFIGURATION - Modifie la valeur du paramètre pour tous les channels \\
    \hline
\end{supertabular}

\subsubsection{Précisions sur l'application du volume}

Lorsque vous changez le volume sur le channel, \FAT calcule un nouveau volume à appliquer sur chaque note.
\medskip

\begin{itemize}
    \item{si la valeur est "df" dans l'écran LIVE, alors le volume de l'instrument ou celui d'une éventuelle commande est appliqué. "df" = "defined"}
    \item{si la valeur est autre que "df", alors le volume est une moyenne calculée entre la valeur écrite dans l'instrument (ou dans une commande) et celle demandée dans le LIVE}
\end{itemize}
\medskip

Un volume de 00 n'éteint donc pas forcément le channel ! Il le divise par 2.
Il n'y a pour le moment pas de mode MUTE.

\ColoredAnnotation{Cette méthode d'application du volume sera peut-être ammené à changer. Si vous avez un avis, faites le moi savoir.}

\subsubsection{Précisions sur l'application du transpose}

Comme pour le volume, \FAT calcul la valeur de transpose à appliquer en fonction d'autres paramètres.
La valeur de transpose à prendre en considération se calcule très simplement comme suit :

\ColoredAnnotation{transpose du projet + transpose sur le block + transpose sur le live = transpose}
