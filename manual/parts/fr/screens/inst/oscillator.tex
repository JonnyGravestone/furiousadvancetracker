\FAT est capable de générer le son simple d'un oscillateur. Ces instruments seront surement utilisés un jour
pour la synthèse FM. En attendant, ils restent utilisables en tant que tel dans les canaux SNA et SNB.

\ColoredAnnotation {Vous remarquerez assez rapidement que les oscillateurs de type sinusoidal n'ont pas un son "parfait".
Cela est dû à 2 choses. La première étant que le processeur de la GBA ne gère pas le calcul flottant (nombres avec virgules).
Le calcul d'une onde sinusoidale demande une certaine précision que l'on ne peut donc pas obtenir nativement et facilement.
La seconde chose : il reste néanmoins possible de faire du calcul flottant, mais de manière logicielle. Ce genre de calcul n'est cependant pas
encore utilisé dans \FAT. Il y a donc encore une marge de progression possible pour ce sujet.}

\Image{images/screen_instrument_oscillator}{0.5}{Type OSCILLATOR}

\subsubsection{Paramètres}

\paragraph{Oscillator : "OSC Shape"} La forme d'onde à générer. \FAT gère 4 types d'ondes :\medskip

\begin{itemize}
    \item{\SimpleImage{images/osc_shape_sinus}{1.0} sinus}
    \item{\SimpleImage{images/osc_shape_square}{1.0} carré}
    \item{\SimpleImage{images/osc_shape_triangle}{1.0} triangle}
    \item{\SimpleImage{images/osc_shape_sawtooth}{1.0} dents de scie}
\end{itemize}\medskip

\paragraph{Oscillator : "Amplitude"} Ce paramètre n'est pas encore disponible à l'utilisation.

\paragraph{Enveloppe : "Volume"} Volume affecté au sample : 50\% (0) ou 100\% (1).

\paragraph{Paramètre : "Timed"} Définit si le son sera coupé ou non.

\paragraph{Paramètre : "Length"} Si "Timed" est à 1, cette valeur définit la coupure en pourcentage de la durée totale du son.

\paragraph{Paramètre : "Output"} Choisissez le mode de sortie du son.
\medskip

\begin{itemize}
    \item{L: le son sort à gauche (LEFT)}
    \item{R: le son sort à droite (RIGHT)}
    \item{RL: le son sort à droite et à gauche (LEFT/RIGHT)}
    \item{VIDE: le son ne sort plus (muet)}
\end{itemize}
