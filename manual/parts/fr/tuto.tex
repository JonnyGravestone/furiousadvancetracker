Cette section va tenter de vous donner quelques premières valeurs de paramètres afin de créer des instruments "intéressants"
et de vous lancer plus rapidement dans la composition de morceaux.
Pour chaque exemple, je vous donnerai la marche à suivre la plus complète possible.

\subsection{Kick "Boom"}

Le kick "boom" se programme sur le canal 1 exclusivement. Il s'agit d'une façon d'obtenir un kick.
\medskip

\begin{itemize}
\item{dans l'écran SONG, placez le curseur sur le canal 1 (PU1).}
\item{appuyez sur L+A afin de poser un nouveau numéro de séquence.}
\item{pressez SELECT, restez appuyé et sélectionnez l'écran BLOCKS.}
\item{une fois dans l'écran BLOCKS, appuyez de nouveau sur L+A afin de faire apparaitre un numéro de block disponible.}
\item{pressez SELECT, restez appuyé et séléctionnez l'écran NOTES.}
\item{dans l'écran NOTES, appuyez sur A, laissez enfoncé et sélectionnez la note "C 6" (ou "C 7", "C 8") avec le curseur.}
\item{editez les paramètres de l'instrument. Appuyez sur SELECT, laissez enfoncé et sélectionnez l'écran INSTRUMENTS.}
\item{l'instrument doit être de type PULSE. Si ca n'est pas le cas, changez le type en pressant L et en allant vers la gauche avec le curseur.}
\item{boostez le volume à fond : F.}
\item{le waveduty doit être positionné sur la valeur 2.}
\item{positionnez la valeur du sweep sur 41.}
\item{appuyez sur START !}
\end{itemize}\medskip

Alternatives:
\medskip

\begin{itemize}
\item{en "timant" l'instrument, le kick sera plus sec: "Timed = 1". Modifiez ensuite la valeur du "SoundLength".}
\item{essayez de changer le sweep avec la valeur 31 ou 51. Une valeur de 32 ou 52 donne encore des résultats différents.}
\item{l'octave de la note est également important ! Essayez de descendre à l'octave 5 et le kick sera moins puissant. A l'inverse, l'octave 8 permet de le faire claquer plus fort.}
\end{itemize}\medskip

Screenshot instrument:
\Image{images/kickboom}{1.0}{Instrument - Kickboom}

\subsection{Le sample stéréo}

Par défaut un sample est mono. Mais il est cependant possible de diriger la lecture d'un son vers la droite ou la gauche avec \FAT.
Supposons que vous souhaitiez créer un sample stéréo.
Il suffit pour cela d'ajouter 2 samples dans la ROM (un pour la partie droite et l'autre pour la gauche) puis de configurer 2 instruments
avec le paramètre OUTPUT réglé de façon judicieuse
 (ou d'utiliser la commande OU, ce qui vous économise un instrument).
Le channel SNA s'occupe de lire celui à gauche avec un OUtput 'Left'.
Le SNB s'occupe de lire celui de droite avec un OUtput 'Right'. L'inverse est également possible. Peu importe.
Le tour est joué ! Vous avez un sample stéréo.
