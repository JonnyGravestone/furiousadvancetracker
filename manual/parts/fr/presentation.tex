\subsection{Présentation et avertissement}
\FAT est un logiciel musical conçu pour la Gameboy Advance.
Avec \FAT, vous pouvez composer de la musique directement sur votre console de jeu en utilisant ses capacités techniques.
\FAT se veut le plus simple possible mais un apprentissage de son mode de fonctionnement reste néanmoins nécessaire avant de pouvoir en exploiter les possibilités.\medskip

La \fatversion  (celle qui est actuellement disponible) est une version stable.
Même si je me suis attaché à éliminer un nombre maximal de problèmes en tout genre, le logiciel est encore très probablement buggé et/ou non finalisé.
Paris ne s'est pas construit en un jour: \FAT va continuer d'évoluer tant que le coeur m'en dit.
Dans le cas où vous détecteriez un bug ou un comportement "bizarre", n'hésitez pas !
Faites un rapport de bug sur \url{http://wwww.furiousadvancetracker.com}.
Le rapport ne vous prendra pas plus de 5 minutes et cela m'aidera pour faire avancer le logiciel !
D'avance merci ! (et un bisou aussi !)
\medskip

Ceux qui ont déjà un pied dans la composition de musique chiptune ne tarderont pas à se rendre compte que l'interface de \FAT ressemble énormément à celle d'LSDJ (LittleSoundDJ).
J'ai en effet considéré LSDJ comme {\bf LE} modèle mais en y ajoutant ma touche personnelle.
Quelques fonctionnalités sont aussi complétement originales, d'autres ont été revues et d'autres encore ont disparues.
\medskip

Je n'ai évidemment pas la prétention d'affirmer que \FAT joue dans la même cour qu'LSDJ.
Néanmoins, j'espère qu'un jour, \FAT deviendra une référence pour la composition de musique chiptune sur GBA.\medskip

Et donc ... un grand merci à toi qui lit ce manuel probablement pour essayer \FAT. J'espère qu'il te plaira !

\subsection{Pourquoi \FAT ?}
A l'époque de la création du projet, début 2011, il existe déjà quelques solutions pour composer de la musique sur une Gameboy Advance:
\medskip

\begin{itemize}
  \item{Nanoloop 2.3/2.5}
  \item{M4G tracker}
  \item{et probablement d'autres que je n'ai su trouver...}
\end{itemize}\medskip

Toutes ces solutions ne me satisfont pas (quoiqu'elles soient toutes parfaitement fonctionnelles!).
\medskip

\begin{itemize}
  \item{Nanoloop 2.3/2.5 est un excellent logiciel ! mais sa disponibilité quasi nulle et son prix en fait une solution finalement pas si simple d'accès.}
  \item{M4G tracker est aussi un excellent logiciel: mais son développement parait être abondonné, tout comme l'était \FAT à une époque cela dit ...}
  \item{On peut faire tourner LSDJ sur une GBA mais l'intérêt est relativement limité.}
  \item{quand aux possibles autres solutions, je ne les ai pas trouvées...}
\end{itemize}\medskip

Et puis, finalement, programmer son propre tracker c'est quand même super fun !
\medskip

Notez que nous sommes en 2017 et la situation n'a pas bougée. Peut-être que tout le monde s'en tamponne de la GBA ? :P \medskip

\subsection{La version de \FAT}

\FAT est versionné avec 3 numéros.
\begin{itemize}
  \item{le premier numéro est le numéro de version majeure. Un changement de numéro implique qu'il y a de grosses incompatibilités avec les versions précédentes.
              Actuellement, la version est la "1"}
  \item{le numéro du milieu est le numéro de version mineure.
              Ce chiffre donne une idée de la progression des features. La compatibilité peut être conservée (mais pas forcément).}
  \item{le dernier numéro est le numéro de version des bugfixes. Il augmentera à chaque correction de bug.}
\end{itemize}
\medskip

La version actuelle est "\fatversion".

\subsection{Comment utiliser \FAT ?}

\FAT est livré sous la forme d'un fichier gba (rom).
Ce fichier est directement utilisable sur votre ordinateur ou smartphone avec l'aide d'un émulateur (VisualBoyAdvance par exemple).
\medskip

Mais un son émulé n'égalera jamais l'original.
Pour exécuter \FAT sur une vraie Gameboy Advance, il vous faudra:
\begin{itemize}
  \item{une GBA. Le meilleur choix reste la Gameboy Micro. L'écran de cette dernière est de meilleure qualité bien que certains le trouvent trop petit. Au pire, la GBA SP est pas mal. Je déconseille le modèle GBA classique (à cause de la position inconfortable du bouton SELECT).}
  \item{une cartouche EZ-Flash, SuperCardToSD, ou autre ... Il faut un peu chercher sur le web ...}
  \item{une carte SD qui va bien avec la cartouche (si vous optez pour une cartouche basée sur SD)}
\end{itemize}\medskip

Après avoir investi dans tout ce petit matériel, il suffira de copier le fichier fat.gba {\bf (à l'aide du logiciel fourni avec votre cartouche)} sur la carte SD: direction la GBA !
Attention ! Veillez à bien suivre les instructions fournis par le constructeur de la cartouche : sans cela, \FAT risque de ne pas fonctionner correctement.

\ColoredAnnotation{Attention ! Bien que stable, cette version de \FAT reste relativement expérimentale.
Il est, je pense, inutile d'investir trop d'argent dans le but de tester sur GBA. C'est vous qui voyez !}

Si vous tenez à acheter votre propre cartouche, alors référez vous au tableau ci-dessous, il devrait vous aider à faire votre choix.

\subsection{Tableau comparatif des cartouches pour GameboyAdvance}

Il existe un certain nombres de cartouches permettant d'exécuter des programmes dits "homebrew" (amateur) ... comme \FAT !
Voici les quelques cartouches déjà testées avec \FAT :
\medskip

% Configuration du tableau
%\tablefirsthead{\hline \rowcolor{blue} {\bf Cartouche} & {\bf "Gravure"} & {\bf Démarrage} & {\bf Sauvegarde} \\ \hline}
\tablehead{\hline \rowcolor{headertab} {\bf Cartouche} & {\bf "Gravure"} & {\bf Démarrage} & {\bf Sauvegarde} \\ \hline}
\tabletail{\hline \multicolumn{4}{|l|} {\small ...} \\ \hline}
\tablelasttail{\hline}
\begin{supertabular}{|p{3.5cm}|l|l|l|}
  \hline
      \begin{minipage}[c]{3cm}
      \vspace{0.5cm}
      \SimpleImage{images/supercardsd_djangofeet_sd}{0.2}
      \end{minipage} &
      \begin{minipage}{3cm}
      Utilisez le logiciel officiel
      \end{minipage} &
      \begin{minipage}{2cm}
      \textcolor{vert}{OK}
      \end{minipage} &
      \begin{minipage}{7cm}
      La sauvegarde fonctionne mais il faut faire une manipulation supplémentaire: après avoir sauvegardé {\bf pressez R+L+A+START et validez}
      \end{minipage} \\
  \hline
      \begin{minipage}[c]{3cm}
      \vspace{0.5cm}
      \SimpleImage{images/supercardminisd_djangofeet_sd}{0.2}
      \end{minipage} &
      \begin{minipage}{3cm}
      Utilisez le logiciel officiel
      \end{minipage} &
      \begin{minipage}{2cm}
      \textcolor{bleu}{NON TESTE}
      \end{minipage} &
      \begin{minipage}{7cm}
      Non testé
      \end{minipage} \\
  \hline
      \begin{minipage}[c]{3cm}
      \vspace{0.5cm}
      \SimpleImage{images/ezflash1}{0.2}
      \end{minipage} &
      \begin{minipage}{3cm}
      Utilisez le logiciel officiel (et son linker USB)
      \end{minipage} &
      \begin{minipage}{2cm}
      \textcolor{vert}{OK}
      \end{minipage} &
      \begin{minipage}{7cm}
      Rien à signaler. Tout fonctionne ! C'est le meilleur choix.
      \end{minipage} \\
  \hline
      \begin{minipage}[c]{3cm}
      \vspace{0.5cm}
      \SimpleImage{images/ultraflash2advance}{0.2}
      \end{minipage} &
      \begin{minipage}{3cm}
      Utilisez les drivers officiels (la GBA fait office de linker)
      \end{minipage} &
      \begin{minipage}{2cm}
      \textcolor{vert}{OK}
      \end{minipage} &
      \begin{minipage}{7cm}
      La sauvegarde fonctionne.
      \end{minipage} \\
  \hline
      \begin{minipage}[c]{3cm}
      \vspace{0.5cm}
      \SimpleImage{images/supercardsd_sd}{0.2}
      \end{minipage} &
      \begin{minipage}{3cm}
      ???
      \end{minipage} &
      \begin{minipage}{2cm}
      \textcolor{rouge}{KO}
      \end{minipage} &
      \begin{minipage}{7cm}
      Non testé. Certaines personnes rapportent que cette cartouche ne permet pas de faire tourner \FAT :(
      \end{minipage} \\
  \hline
      \begin{minipage}[c]{3cm}
      \vspace{0.5cm}
      \SimpleImage{images/ezflash4}{0.2}
      \end{minipage} &
      \begin{minipage}{3cm}
      \FAT démarre et est utilisable. Attention ! Il est {\bf nécessaire} d'utiliser le logiciel fourni !
      \end{minipage} &
      \begin{minipage}{2cm}
      \textcolor{vert}{OK}
      \end{minipage} &
      \begin{minipage}{7cm}
      La sauvegarde ne fonctionne pas ! Utilisez celle livrée avec la cartouche (la save cheat).
      \end{minipage} \\
  \hline
      \begin{minipage}[c]{3cm}
      \vspace{0.5cm}
      Sur smartphone
      \SimpleImage{images/smartphone}{0.5}
      \end{minipage} &
      \begin{minipage}{3cm}
      \FAT démarre et est utilisable.
      \end{minipage} &
      \begin{minipage}{2cm}
      \textcolor{vert}{OK}
      \end{minipage} &
      \begin{minipage}{7cm}
      Consultez votre store pour trouver un émulateur fonctionnel.
      \end{minipage} \\

\end{supertabular}

\medskip Le meilleur choix reste l'ezflash1. A chercher sur la toile.

\medskip Si vous utilisez l'EZFlashIV et que vous avez un souci de curseur "noir", voici la solution trouvée par \textcolor{bleu}{\bf @maz\_hoot} (sur Twitter, merci pour le partage ! ) :
\ColoredAnnotation{Il faut écrire FAT comme jeu en chargement rapide. Pour ce faire, quand on est sur l'écran de lancement de la rom FAT, appuyez sur select : un panneau s'ouvre. Pressez A et ça écrit. Il est ensuite selectionnable en lancement rapide et le curseur est bien vert.}

\subsection{Changements pour la \fatversion}

Liste des changements pour cette version de \FAT: \medskip
\begin{itemize}
  \item{stabilisation générale (réécriture/réorganisation de tout le code)}
  \item{réécriture de la lecture des samples et oscillator}
  \item{ajout d'un editeur de courbe pour les instruments de type WAVE}
  \item{ajout d'un paramètre pour gérer la vitesse d'échantillonage des samples/oscillators}
\end{itemize}

\subsection{Roadmap pour la version suivante}

La version suivante sera la \fatnextversion. \medskip
\begin{itemize}
  \item{ajout de commandes diverses}
  \item{refonte de l'écran de gestion de courbes dans le channel WAV}
  \item{recherche de documentation pour implémenter la synthèse FM (si vous en avez je suis preneur) afin de moduler les samples et les faire sonner différemment.
              L'objectif serait de se rapprocher du son de Nanoloop.}
\end{itemize}

\subsection{Premier son}

Commençons simple ! Nous allons poser notre premier son dans \FAT.\medskip

Lorsque \FAT démarre, c'est l'écran SONG qui apparaît en premier.
Ne bougez pas le curseur, appuyez simplement une première fois sur A.
Un "00" apparait en haut à gauche du tableau: c'est une {\bf séquence}.

\Image{images/screen_song}{1.0}{L'écran SONG}

Pressez ensuite SELECT (laissez la touche enfoncée) et appuyez sur la croix directionnelle vers la droite: l'interface change.

\Image{images/screen_popup}{1.0}{L'écran POPUP}

Sans bouger le curseur, appuyez une fois sur A: un nouveau "00" apparait.
Cette fois-ci il s'agit d'un {\bf block}. La séquence "00" contient donc maintenant un seul block nommé "00".

\Image{images/screen_blocks}{1.0}{L'écran BLOCKS}

Pressez à nouveau SELECT (restez appuyé) et déplacez vous encore d'un cran vers la droite.
L'interface a encore changé !
Appuyez une dernière fois sur A : un "C 3" apparait.
C'est votre première {\bf note} !
Appuyez sur START : vous avez joué votre premier son avec \FAT sur une Gameboy Advance !

\Image{images/screen_notes}{1.0}{L'écran NOTES - appuyez sur START !}

\subsection{Structure des données}

Nous avons abordé les notions de "séquences", "blocks" et de "notes".
\FAT est globalement organisé de la même façon qu'LSDJ:
\medskip

\begin{itemize}
  \item{un tableau de séquences sur 6 colonnes: la GBA sait jouer 6 sons en même temps.}
  \item{dans chaque séquence, un nombre défini de blocks (16 maximum)}
  \item{dans chaque block, les notes ! (16 maximum)}
\end{itemize}
\medskip

La façon d'organiser une chanson ressemble un peu au principe des "poupées russes".
Les séquences contiennent des blocks qui eux même embarquent les notes.
Quand \FAT lit une séquence, il passe d'un block à l'autre et lit toutes les notes une par une.
\medskip

Pour éviter que chaque note sonne de la même façon, on leur assigne un {\bf instrument}.
Chaque instrument contient un certain nombre de paramètres.
Si vous changez les valeurs de ces paramètres, alors la note sonnera différemment.
Il faut également savoir qu'il existe plusieurs types d'instruments : les canaux sonores de la GBA permettent de faire des choses différentes sur chacun d'eux.

\subsection{Données techniques de la Gameboy Advance}

La GameboyAdvance fournit 4 canaux "analogiques" pour la génération de sons et de bruits.
Ces canaux sont sensiblement les mêmes que ceux présents dans la Gameboy première du nom.
La GBA propose en plus 2 canaux "numériques": ils peuvent être utilisés afin de jouer des sons plus compliqués.
Attention, la Gameboy ne "génère" plus le son en passant par ces 2 derniers canaux: on s'éloigne ainsi de l'esprit "chiptune".

La GBA ne possède qu'un haut-parleur mono.
Si vous souhaitez profiter des possibilités stéréo de \FAT, il vous faudra brancher un casque sur la sortie prévue à cet effet.
La GBA ne sort du stéréo que via cette prise.

\paragraph{Canal 1 : PU1} Le canal permet de jouer des mélodies et/ou d'appliquer un effet SWEEP sur les notes. Attention, seul le canal 1 en est capable !

\ColoredAnnotation{Vous pouvez utiliser ce canal pour générer un kick avec le sweep, ou comme basse principale}

\paragraph{Canal 2 : PU2} Le canal 2 est exactement le même que le 1 à la différence qu'il n'est pas capable de gérer un sweep.

\ColoredAnnotation{Vous pouvez utiliser ce canal pour jouer une mélodie.}

\paragraph{Canal 3 : WAVE} Le canal 3 est capable de lire des minis-samples répartis dans 2 banques.

\ColoredAnnotation{Ce canal peut servir de soutien à la basse, ou faire office de "guitare electrique" et ajouter un côté saturé à votre morceau.}

\paragraph{Canal 4 : NOISE} Ce canal est spécialisé dans la génération de bruits blancs.

\ColoredAnnotation{Utilisez ce canal comme soutien au kick, ou pour des effets spéciaux (comme le son d'une plage en été par exemple, ...)}

\paragraph{Canal 5 : SOUNDA} Ce canal sait jouer des samples et du son plus proche des machines actuelles que du "vrai" chiptune.

\paragraph{Canal 6 : SOUNDB} Ce canal sait jouer des samples et du son plus proche des machines actuelles que du "vrai" chiptune.

\ColoredAnnotation{Ces deux canaux sont parfaits pour ajouter de la voix à vos tracks ou des kicks plus puissants/ronds/travaillés.
                  Vous verrez qu'il est possible d'insérer {\bf vos} samples dans la rom pour utilisation directe dans \FAT.}

\subsubsection{Les différences avec la Gameboy}

Le contrôleur audio de la GBA est essentiellement le même que celui de la Gameboy monochrome/couleur.
Voici les quelques changements opérés:

\paragraph{SNA et SNB} Deux canaux sonores ont été ajoutés qui peuvent être utilisés pour jouer du son numérique.
                      Le débit de données de l'échantillon et l'échantillon lui-même doit être fourni et géré à l'aide d'une minuterie et/ou un canal DMA.

\paragraph{Registres} Le registre SOUNDCNT\_H contrôle les nouveaux canaux DMA - ainsi que le mélange avec les quatre anciens canaux.
                      Le registre SOUNDBIAS contrôle la sortie finale du son.

\paragraph{Amélioration du canal WAV} La longueur de la RAM Wave est doublée : 2 banques de 32 chiffres chacune.
                      Il est possible de jouer les 2 banques (l'une après l'autre) ou de n'en sélectionner qu'une.

\medskip
Si vous souhaitez en savoir plus sur les spécifications techniques de la Gameboy Advance,
vous pouvez consulter \href{http://www.portabledev.com/media/GBA/tutoriels/gbatek.htm}{www.portabledev.com}.
Le site \href{http://belogic.com/gba/}{belogic.com/gba/} est également une sacrée mine de renseignements !
