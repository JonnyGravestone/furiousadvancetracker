\label{sec:addsamples}

A partir de la version 1.0.0, il est possible d'ajouter vos propres samples à \FAT.
Cette manipulation (technique) est possible sur le site web (et beaucoup plus simple, le programme sur le site se charge de tout)
mais possibité vous est offerte de le faire vous même sur votre poste si le coeur vous en dit. Et que vous n'avez rien de mieux à faire.
Voyons la marche à suivre.
\medskip

\begin{itemize}
  \item{réunissez vos samples dans un dossier quelconque}
  \item{créez un fichier 0infos et saisissez le nom de votre collection : "my-first" par exemple (8 caractères obligatoires !) }
  \item{transformez vos fichiers samples dans le bon format}
  \item{créez le fichier GBFS qui sera intégré à la ROM. Répétez l'opération pour chaque kit.}
  \item{dans un autre dossier, créez le fichier 00nbkit qui compte le nombre de kits dans la ROM finale (le nombre de gbfs donc)}
  \item{intégrez les fichiers GBFS dans la ROM "défault"}
\end{itemize}\medskip

Pas d'inquiétudes ! Je vais tout expliquer !
Commencez par télécharger la ROM "défault" sur le site dédié.

\paragraph{Format} La première chose à savoir est le format audio cible : {\bf pcm\_s8, 16000Hz, mono}.
Ce format est très important ! Si vous ne le respectez pas, vos samples ne sonneront pas correctement.
Au mieux vous aurez du bruit, au pire vous planterez sytématiquement votre GBA.

\paragraph{Fichier de description} Une collection de samples est décrite par un fichier toujours nommé {\bf 0infos}.
Le contenu de ce fichier texte est très simple : les 8 premières lettres sur la première ligne sont le nom de votre collection.
Le reste est ignoré. Assurez vous de respecter ces 8 lettres (quitte à mettre des '-').

\paragraph{Convertir les fichiers audio} Libre à vous d'utiliser l'outil que vous souhaitez.
Pour ma part, j'utilise "ffmpeg" qui a le mérite de se lancer en ligne de commande et est donc scriptable.
Voici le script que j'utilise pour transformer une collection complète de MP3 dans le bon format :
\Annotation{find . -name "*.mp3" -exec ffmpeg -i {} -ac 1 -ar 16000 -f s8 -acodec pcm\_s8 \{\}.snd \textbackslash;}
Audacity ou tout autre logiciel de traitement audio est également approprié.
L'important est que le format soit respecté.
Par convenance, nommez vos samples formatés avec l'extension ".snd".

\paragraph{Le fichier GBFS} Ce fichier est le conteneur qui va embarquer le fichier description suivi de tous vos fichier ".snd".
Pour le créer, il faut utiliser l'utilitaire "gbfs.exe".
Vous pouvez télécharger ce fichier sur le Github de \FAT (\href{https://github.com/cbrouillard/furiousadvancetracker/blob/master/gbfs.exe?raw=true}{Github - gbfs.exe})
\Annotation{gbfs.exe monfichier\_sample.gbfs samples/0infos samples/*.snd}
Le fichier "monfichier\_sample.gbfs" sera créé.
L'ordre de cette commande est important : le fichier 0infos DOIT être le premier.
{\bf Cet utilitaire fonctionne sous Linux avec Wine}.

\paragraph{Le fichier 00nbkit} Lorsque vous avez terminé de créer vos fichiers GBFS pour chaque kit de sample, un fichier 00nbkit doit être créé.
Une seule donnée à l'intérieur de ce fichier : le nombre de kits que vous allez intégrer dans la ROM.
Si vous n'avez qu'un seul kit à intégrer :
\Annotation{echo 1 \textgreater  00nbkit}
Une fois ce fichier créé, il faut aussi le tranformer en GBFS.
\Annotation{gbfs.exe 00nbkit.gbfs 00nbkit}

\paragraph{Intégration dans la ROM} Maintenant que vos fichier GBFS et 00nbkit.gbfs sont créés, il suffit de les ajouter à la ROM.
Et c'est aussi simple que cette commande :
\Annotation{cat FAT\_default.gba 00nbkit.gbfs monfichier\_sample.gbfs \textgreater FAT\_v1.0.0\_withsample.gba}
Oui. Un simple "cat".
Attention cependant, le fichier 00nbkit.gbfs {\bf DOIT} être le premier.
(\FAT se sert de la donnée écrite dans le fichier 00nbkit pour connaitre à l'avance le nombre de kit)
\medskip

Après toutes ces manipulations, votre ROM est prête : vos samples sont accessibles sur les channels SNA et SNB avec des instruments de type SAMPLE. Enjoy !
