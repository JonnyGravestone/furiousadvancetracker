\ColoredAnnotation{\FAT s'enrichit de commandes au fur et à mesure de sa conception. Toutes ne sont pas encore présentes mais ce n'est qu'une question de temps ! Patience :)}

Voici les commandes actuellement utilisables.

\subsection{CH (CHord)}

\ColoredAnnotation {Fonctionne sur les canaux PU1, PU2, WAVE et NOISE}

Génére un arpège constitué de 3 notes. Ex: "C 4 CH3C" donne un arpège de "0,3,C,0,3,C, etc ...".

\subsection{CV (Custom Voice)}

\ColoredAnnotation{Fonctionne sur le canal WAVE uniquement}

Sélectionne une onde personalisée à appliquer sur la note.

\subsection{DL (DeLay)}

\ColoredAnnotation{Fonctionne sur tous les canaux}

Joue la note avec un délai.

\Annotation{\textcolor{red}{Attention, jouer un autre effet ou une autre note avant la durée effective du délai annule celui-ci. \FAT n'est pas capable de paralléliser la gestion du délai par rapport aux autres événements que constituent les notes ou les commandes.}}

\subsection{HO (Hop!)}

\ColoredAnnotation{Fonctionne sur tous les canaux}

Cette commande permet de sauter directement au block suivant.
Elle peut permettre de créer des décalages de temps entre les blocks.
La valeur que vous paramétrez correspond au numéro de ligne vers lequel sauter.

\subsection{KL (KilL)}

\ColoredAnnotation{Fonctionne sur tous les canaux}

Cette commande permet de stopper (kill = tuer) le son. Il est possible de donner un délai.

\subsection{PI (PItch)}

\ColoredAnnotation{Fonctionne sur les canaux PU1, PU2 et WAVE}

Modifie le pitch d'une note.

\Annotation{Cet effet est encore à l'état expérimental.}

\subsection{OU (OUtput)}

\ColoredAnnotation{Fonctionne sur tous les canaux}

Dirige le son vers la droite (\_R), gauche (L\_), les deux (LR) ou aucun des deux (\_\_)

\subsection{RT (ReTrig)}

\ColoredAnnotation{Fonctionne sur tous les canaux}

Rejoue la même note avec les mêmes paramètres. La valeur de l'effet indique la rapidité de répétition.

\subsection{SL (SLide)}

\ColoredAnnotation{Fonctionne sur les canaux PU1, PU2 et WAVE}

Applique un slide d'une note vers une autre.

\subsection{SR (SampleRate)}

\ColoredAnnotation{Fonctionne sur tous les canaux mais n'a d'impact que sur les canaux SNA et SNB}

Modifie le paramètre projet "SampleRate". Ce paramètre modifie la vitesse d'échantillonage pour les instruments de type SAMPLE ou OSCILLATOR.

\Annotation{\textcolor{red}{Attention, le paramètre "SampleRate" s'applique pour TOUTE la track (et pas seulement la note concernée par la commande)}}

\subsection{SW (SWeep)}

\ColoredAnnotation{Fonctionne sur le canal PU1 uniquement}

Cette commande n'est effective que sur le channel 1 (PU1) : elle permet de modifier la valeur du sweep.
Ce sweep remplace celui configuré dans l'instrument.

\subsection{TP (TemPo)}

\ColoredAnnotation{Fonctionne sur tous les canaux}

Modifie le paramètre projet "Tempo".

\subsection{TS (TranSpose)}

\ColoredAnnotation{Fonctionne sur les canaux PU1, PU2 et WAVE}

Modifie le transpose d'une note. Ce transpose s'ajoute à celui du projet, du block et de l'instrument.

\subsection{VB (ViBrato)}

\ColoredAnnotation{Fonctionne sur les canaux PU1, PU2 et WAVE}

Applique un effet de vibrato sur la note. La valeur indique la "puissance" du vibrato.

\subsection{VO (VOlume)}

\ColoredAnnotation{Fonctionne sur tous les canaux}

Cette commande permet de modifier le volume du son.

\subsection{WA (WAveform)}

\ColoredAnnotation{Fonctionne sur les canaux PU1 et PU2 et sur les canaux SNA et SNB pour les instruments de type OSCILLATOR}

Modifie la forme d'onde à appliquer à la note.
