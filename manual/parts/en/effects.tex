\ColoredAnnotation{\FAT is not yet rich in effects of any kind for the moment: it will have to be content for the time.
Several commands are in the program and will land in the following versions. Patience!}

Here are the currently usable commands.

\subsection{CH (Chord)}

\ColoredAnnotation {Works on PU1, PU2, WAVE and NOISE}

Generates an arpeggio consisting of 3 notes. Ex: "C 4 CH3C" give "0,3,C,0,3,C, etc ..." arpeggio.

\subsection{HO (Hop!)}

This command lets you jump directly to the next block.
It can create time lags between blocks.
The value you set is the line number to jump to.

\subsection{KL (KilL)}

This command is used to stop the sound.

\ColoredAnnotation{No value is currently available for this parameter.}

\subsection{OU (OUtput)}

Direct the sound to the right (\_R), left (L\_), both (LR) or neither (\_\_)

\subsection{SR (SampleRate)}

Change project's parameter "SampleRate" value.

\subsection{SW (SWeep)}

This command is effective only on channel 1 (PU1): it allows to modify the value of the sweep.
This sweep replaces the one configured in the instrument.

\subsection{TP (TemPo)}

Change project's parameter "Tempo".

\subsection{VO (VOlume)}

Use this command to change the volume of the sound.
\#captainobvious.
Note that depending on the channel, the value you pass is taken differently.
