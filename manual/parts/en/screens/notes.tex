C'est dans cet écran que vous allez écrire ... vos notes !
Vous ne pouvez écrire qu'une seule note par ligne.
Un block possède 16 lignes, vous pouvez donc y insérer 16 notes.
\medskip

La notation anglo-saxonne est utilisée pour l'affichage des notes.
Ainsi, le "DO" s'écrit "C" chez nos amis anglais.
Voici un tableau de correspondance :
\medskip

\begin{center}
% Configuration du tableau
    \tablehead{\hline \rowcolor{headertab} {\bf FR} & {\bf EN} \\ \hline}
    \tabletail{\hline \multicolumn{2}{|l|} {\small ...} \\ \hline}
    \tablelasttail{\hline}
    \begin{supertabular}{|l|l|}
    \hline
        {\bf DO} & C \\
        \hline
        {\bf RE} & D \\
        \hline
        {\bf MI} & E \\
        \hline
        {\bf FA} & F \\
        \hline
        {\bf SOL} & G \\
        \hline
        {\bf LA} & A \\
        \hline
        {\bf SI} & B \\
    \hline
    \end{supertabular}
\end{center}


Chaque note est reliée à un instrument (la colonne INST).
Chaque instrument possède un certains nombres de paramètres permettant de faire sonner une note différemment.
Nous verrons dans la section suivante comment paramétrer ces instruments.

\Image{images/screen_notes_complete}{1.0}{L'écran NOTES}

\subsubsection{Commandes}
% Configuration du tableau
\tablehead{\hline \rowcolor{headertab} {\bf Touche(s)} & {\bf Effet} \\ \hline}
\tabletail{\hline \multicolumn{2}{|l|} {\small ...} \\ \hline}
\tablelasttail{\hline}
\begin{tabular}{|l|p{11cm}|}
\hline
    \rowcolor{headertab} {\bf Touche(s)} & {\bf Effet} \\
    \hline
    {\bf SELECT} & Passer en mode popup \\
    \hline
    {\bf R+L} & Afficher l'écran d'aide \\
    \hline
    {\bf START} & Lancer/stopper la lecture des notes \\
    \hline
    {\bf L+HAUT} & Déplacer le curseur tout en haut \\
    \hline
    {\bf L+BAS} & Déplacer le curseur tout en bas \\
    \hline
    {\bf A} & Inscrire une note (par défaut "C 3") \\
    \hline
    {\bf A+HAUT/BAS} & COLONNE NOTE | Modifier l'octave de la note désignée par le curseur \\
    \hline
    {\bf A+DROITE/GAUCHE} & COLONNE NOTE | Modifier la tonalitée de la note désignée \\
    \hline
    {\bf A+DIRECTION} & COLONNE INSTRUMENT | Modifier le numéro d'instrument utilisée pour la note désignée \\
    \hline
    {\bf A+DIRECTION} & COLONNE CMD | Modifier les valeurs de commandes \\
    \hline
    {\bf L+A} & COLONNE INSTRUMENT | Trouver un nouveau numéro d'instrument et l'inscrire \\
    \hline
    %{\bf R+A} & Non utilisé pour le moment \\
    %\hline
    {\bf R+DROITE/GAUCHE} & Modifier le numéro de block en cours d'édition \\
    \hline
    %{\bf R+HAUT/BAS} & Sauter au block précédent/suivant dans la séquence en cours d'édition \\
    %\hline
    {\bf B} & SI CASE NON VIDE | Couper la note \\
    \hline
    {\bf B} & SI CASE VIDE | Coller la note préalablement coupée \\
\hline
\end{tabular}
