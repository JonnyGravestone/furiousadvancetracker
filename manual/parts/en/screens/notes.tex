With this screen, you'll be able to write notes for your tracks !
Only one note is allowed for a line and a block can contains up to 16 notes.
\medskip

Each note has an assigned instrument (INST column) which setup some parameters in order to make your note sound differently each other.
We'll see in the next section how to setup instruments.

\Image{images/screen_notes_complete}{1.0}{NOTES screen}

\subsubsection{Commands}
% Configuration du tableau
\tablehead{\hline \rowcolor{headertab} {\bf Key(s)} & {\bf Effect} \\ \hline}
\tabletail{\hline \multicolumn{2}{|l|} {\small ...} \\ \hline}
\tablelasttail{\hline}
\begin{tabular}{|l|p{11cm}|}
\hline
    \rowcolor{headertab} {\bf Key(s)} & {\bf Effect} \\
    \hline
    {\bf SELECT} & Show the switch popup \\
    \hline
    {\bf R+L} & Show the help screen \\
    \hline
    {\bf START} & Start/stop notes playback \\
    \hline
    {\bf L+UP} & Move the cursor to top \\
    \hline
    {\bf L+DOWN} & Move the cursor to bottom \\
    \hline
    {\bf A} & Write a new note (default is "C 3") \\
    \hline
    {\bf A+UP/DOWN} & NOTE COLUMN | Change note's octave under the cursor \\
    \hline
    {\bf A+LEFT/RIGHT} & NOTE COLUMN | Change note's tonality under the cursor \\
    \hline
    {\bf A+DIRECTION} & INSTRUMENT COLUMN | Change the instrument number affected to note \\
    \hline
    {\bf A+DIRECTION} & CMD COLUMN | Change command value \\
    \hline
    {\bf L+A} & INSTRUMENT COLUMN | Find a new free instrument number and set it to spot \\
    \hline
    %{\bf R+A} & Non utilisé pour le moment \\
    %\hline
    {\bf R+LEFT/RIGHT} & Change the currently edited block's number \\
    \hline
    %{\bf R+HAUT/BAS} & Sauter au block précédent/suivant dans la séquence en cours d'édition \\
    %\hline
    {\bf B} & IF NOT EMPTY SPOT | Cut note \\
    \hline
    {\bf B} & IF EMPTY SPOT | Paste note \\
\end{tabular}
