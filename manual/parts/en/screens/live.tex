Here you can play your song as you like.
It is this mode that is to be privileged if you want to wet the shirt on stage.
If you prefer to simply press START in the SONG screen and dance with the audience, free to you!
Note however that the LIVE mode is there to allow you to modify the mechanics of your songs.

\Image{images/screen_live}{1.0}{The LIVE screen appears like the SONG screen}

The sequencer is displayed again but this time there is no question of editing the sequences on the board.
However, you will be able to start each channel in the order that suits you.
Once a channel is started, it is possible to perform several actions :
\medskip

\begin{itemize}
    \item{change channel volume}
    \item{change channel transpose}
    \item{stop channel playback}
\end{itemize}
\medskip

Note that the LIVE screen has two modes: the MANUAL (MAN) mode and the AUTO mode.
In MANUAL mode, FAT does NOT progress in the sequencer. The same sequence on the channel is replayed infinitly: it is up to you to advance the piece.
Conversely, in AUTO mode, FAT behaves as in the SONG screen: all channels currently playing progress normally.
\newpage % hack pourri
\subsubsection{Commands}
\begin{supertabular}{|l|p{10cm}|}
    {\bf SELECT} & Show the switch popup \\
    \hline
    {\bf R+L} & Show the help screen \\
    \hline
    {\bf START} & Stop/Start under cursor's channel playback \\
    \hline
    {\bf R+START} & Stop/Start all channels playback \\
    \hline
    {\bf R+DOWN} & Switch to screen's configuration part (volume change, tsp, tempo and mode) \\
    \hline
    {\bf R+UP} & Switch to screen's sequencer part \\
    \hline
    {\bf A+DIRECTION} & IN CONFIGURATION PART - Change value under cursor \\
    \hline
    {\bf A+L+DIRECTION} & IN CONFIGURATION PART - Change value for all channel at the same time \\
\end{supertabular}

\subsubsection{About volume application}

When you change the volume on the channel, \FAT calculates a new volume to apply to each note.
\medskip

\begin{itemize}
    \item{if the value is "df" in the LIVE screen, then the volume of the instrument or a possible command is applied. "Df" = "defined"}
    \item{if the value is other than "df", then the volume is an average calculated between the value written in the instrument (or within a command) and that requested in the LIVE}
\end{itemize}
\medskip

A volume of 00 does not necessarily set off the channel! It divides it by 2.
There is currently no MUTE mode.

\ColoredAnnotation{This method of volume application may need to change. If you have a notice, let me know.}

\subsubsection{About transpose application}

As for the volume, \FAT calculates the transpose value to be applied according to other parameters.
The transpose value to be taken into consideration is calculated very simply as follows:

\ColoredAnnotation{project transpose + block transpose + live transpose = transpose}
