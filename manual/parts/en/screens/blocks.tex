This screen lets you edit sequence's content : each sequence can contain up to 16 blocks.
When \FAT is going to play this sequence, it will begin by reading the first block and continue to bottom.
When an empty spot has been reached, \FAT will start again at the first line.
In the case of there isn't any blocks in the current sequence, then \FAT will never play it.
\medskip

To make it simple, you can consider this screen a block list contained by a sequence (the sequence's number is displayed on the right top of screen).

\Image{images/screen_blocks_complete}{1.0}{BLOCKS screen}

\subsubsection{Commands}
% Configuration du tableau
\tablehead{\hline \rowcolor{headertab} {\bf Key(s)} & {\bf Effect} \\ \hline}
\tabletail{\hline \multicolumn{2}{|l|} {\small ...} \\ \hline}
\tablelasttail{\hline}
\begin{supertabular}{|l|p{11cm}|}
\hline
    {\bf SELECT} & Show the switch popup \\
    \hline
    {\bf R+L} & Show the help screen \\
    \hline
    {\bf START} & Start/stop blocks playback from currently edited sequence \\
    \hline
    {\bf L+HAUT} & Move the cursor to top \\
    \hline
    {\bf L+BAS} & Move the cursor to bottom \\
    \hline
    {\bf A} & Write last known block's number on the spot designated by cursor (default is "00") \\
    \hline
    {\bf A+DIRECTION} & Modify value \\
    \hline
    {\bf L+A} & Find a new available number and set the value on spot \\
    \hline
    %{\bf R+A} & Non utilisé pour le moment \\
    %\hline
    {\bf B} & IF NOT EMPTY SPOT | Cut block \\
    \hline
    {\bf B} & IF EMPTY SPOT | Paste block \\
    \hline
    {\bf L+B} & IF NOT EMPTY SPOT | Change number of block while copying its content \\
    \hline
    {\bf L+B} & IF EMPTY SPOT | Paste block but changing number \\
    \hline
    {\bf R+DROITE/GAUCHE} & Change the currently edited sequence number \\
    %\hline
    % {\bf R+HAUT/BAS} & Sauter à la séquence précédente/suivante dans le séquenceur \\
\hline
\end{supertabular}
