\FAT is able to generate the simple sound of an oscillator. These instruments will surely be used one day
For the FM synthesis. In the meantime, they remain usable as such in the SNA and SNB channels.

\ColoredAnnotation {You will soon notice that sinusoidal oscillators do not have a "perfect" sound.
This is due to 2 things. The first one is that the GBA processor does not support floating calculation (numbers with commas).
The calculation of a sinusoidal wave requires a certain precision which can not therefore be obtained natively and easily.
The second thing: it is nevertheless possible to do floating calculation, but in a software way. This kind of calculation is not, however,
still used in \FAT. There is therefore still room for progress on this subject.}

\Image{images/screen_instrument_oscillator}{0.5}{OSCILLATOR type}

\subsubsection{Parameters}

\paragraph{Oscillator : "OSC Shape"} The waveform to generate. \FAT supports 4 types of waves :\medskip

\begin{itemize}
    \item{\SimpleImage{images/osc_shape_sinus}{1.0} sinus}
    \item{\SimpleImage{images/osc_shape_square}{1.0} square}
    \item{\SimpleImage{images/osc_shape_triangle}{1.0} triangle}
    \item{\SimpleImage{images/osc_shape_sawtooth}{1.0} sawtooth}
\end{itemize}\medskip

\paragraph{Oscillator : "Amplitude"} This parameter is not yet available for use.

\paragraph{Enveloppe : "Volume"} Volume assigned to the sample: 50\% (0) or 100\% (1).

\paragraph{Parameter : "Timed"} Sets whether the sample will be cut or not.

\paragraph{Parameter : "Length"} If "Timed" is set to 1, this value sets the percentage cut-off of the total sample time.

\paragraph{Parameter : "Output"} Select the sound output mode.
\medskip

\begin{itemize}
    \item{L: the sound goes out to the left (LEFT)}
    \item{R: the sound goes out to the right (RIGHT)}
    \item{RL: the sound goes out to the left and right (LEFT/RIGHT)}
    \item{VIDE: sound does no go out anymore}
\end{itemize}
