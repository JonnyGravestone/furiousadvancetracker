\Image{images/screen_instrument_noise}{0.5}{NOISE type}

\subsubsection{Parameters}

\paragraph{Enveloppe: "Volume"} from 0 to F.
This value adjusts the volume of notes attached to this instrument.
\medskip

\begin{itemize}
    \item{0: muted}
    \item{F: maximum}
\end{itemize}

\paragraph{Enveloppe: "Direction"} \SimpleImage{images/envelope_direction_down}{1.0} or \SimpleImage{images/envelope_direction_up}{1.0}.
Indicates whether the sound goes up or down.

\paragraph{Enveloppe : "Steptime"} from 0 to 7.
Acts on the length of the wave pitch.
\medskip

\paragraph{Enveloppe : "Polystep"} 0 or 1.
Changes the way the noise generator works.
\medskip

\begin{itemize}
    \item{0: simple noise}
    \item{1: squeaking noise}
\end{itemize}

\paragraph{Parameter : "Timed"} 0 or 1.
\medskip

\begin{itemize}
    \item{0: the sound is not timed}
    \item{1: the sound is timed, set the duration with the parameter "Soundlength"}
\end{itemize}

\paragraph{Parameter : "Soundlength"} from 0 to 3F.
This parameter can only be accessed if "Timed" has been set to 1.
\medskip

\begin{itemize}
    \item{0: sound is "infinite"}
    \item{3f: sound is really, really short}
\end{itemize}

\paragraph{Parameter : "Output"} Select the sound output mode.
\medskip

\begin{itemize}
    \item{L: the sound goes out to the left (LEFT)}
    \item{R: the sound goes out to the right (RIGHT)}
    \item{RL: the sound goes out to the left and right (LEFT/RIGHT)}
    \item{VIDE: sound does no go out anymore}
\end{itemize}

\paragraph{Simulator} Change the note written in this space:
         it will be played with the parameters of the instrument you are editing.
