\FAT is able to play samples on the SNA and SNB channels.
A sample is a short-lived soundtrack included in a collection.
A collection series is included by default in \FAT : \medskip

\begin{itemize}
    \item {DEV-MESS : some test samples for development}
    \item {LS-MGDRV : Megadrive samples provided by Little-Scale (\href{http://little-scale.blogspot.fr/2008/08/sega-mega-drive-sample-pack.html}{Little-Scale website})}
    \item {C64DRUMS : drums and snares from Commodore64}
    \item {YM-DRUMS : drums and snares from Atari ST}
    \item {BATTERY : kicks, drums and snares from a "classical" battery}
    \item {RHYTHMS : some complet rhytms (but not exploitable because badly sampled)}
    \item {TR-606-7 : samples of this mythical machine}
\end{itemize} \medskip

It's easy enough to add your own collection of samples. For more informations, please consult \hyperref[sec:addsamples]{"How to manipulate ROM in order to add your own samples kit"} section.

\Image{images/screen_instrument_sample}{0.5}{SAMPLE/KIT type}

\subsubsection{Parameters}

\paragraph{Kit collection : "Name"} The name of the collection of samples in which to draw.

\paragraph{Enveloppe : "Volume"} Volume assigned to the sample: 50\% (0) or 100\% (1).

\paragraph{Parameter : "Loop"} Is the sample played looped?
                                 If "YES", then the sample playback is repeated.

\paragraph{Parameter : "Timed"} Sets whether the sample will be cut or not.

\paragraph{Parameter : "Length"} If "Timed" is set to 1, this value sets the percentage cut-off of the total sample time.

\paragraph{Parameter : "Offset"} Start offset for sample playback.
                                 As a percentage of the total duration of the sample.

\paragraph{Parameter : "Output"} Select the sound output mode.
\medskip

\begin{itemize}
    \item{L: the sound goes out to the left (LEFT)}
    \item{R: the sound goes out to the right (RIGHT)}
    \item{RL: the sound goes out to the left and right (LEFT/RIGHT)}
    \item{VIDE: sound does no go out anymore}
\end{itemize}

\paragraph{Simulator} Change the note written in this space:
         it will be played with the parameters of the instrument you are editing.
