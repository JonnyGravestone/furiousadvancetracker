\label{sec:filesystem}

L'écran FILESYSTEM permet de sauvegarder et de charger les tracks sauvegardées sur la cartouche.
Notez que cet écran dispose de deux modes : SAVE ou LOAD.
Les deux modes sont visuellement très proches: les actions possibles sont néanmoins différentes.

\subsubsection{Mode SAVE}
\Image{images/screen_filesystem_save}{1.0}{L'écran FILESYSTEM en mode SAVE}

En mode SAVE, il est possible de sauvegarder votre projet en cours d'édition.
Sélectionnez le slot sur lequel vous souhaitez sauvegarder et appuyer sur A: un clavier apparait.
Saisissez un nom trop cool pour votre track. Appuyez sur R (ou START) pour confi{\bf R}mer, L (ou SELECT) pour annu{\bf L}er.

\Image{images/keyboard_save}{1.0}{Clavier de saisie}

Une ultime boite de dialogue demande confirmation avant d'effectuer la sauvegarde: appuyez sur R (ou START) pour confi{\bf R}mer, L (ou SELECT) pour annu{\bf L}er.

\Image{images/confirmation_save}{1.0}{Confirmation}

\ColoredAnnotation{\bf \textcolor{red}{Utilisateur de SuperCardSD ! N'oubliez pas de "valider" votre sauvegarde ! Une fois la procédure gérée par \FAT complétée, appuyez sur R+L+A+START. Cela aura pour effet d'écrire le fichier .sav sur votre carte SD. Si vous n'effectuez pas cette manipulation, votre sauvegarde ne sera pas effective et vous risquez de perdre votre travail.} }

\subsubsection{Commandes}
% Configuration du tableau
\tablehead{\hline \rowcolor{headertab} {\bf Touche(s)} & {\bf Effet} \\ \hline}
\tabletail{\hline \multicolumn{2}{|l|} {\small ...} \\ \hline}
\tablelasttail{\hline}
\begin{supertabular}{|l|p{11cm}|}
\hline
    {\bf SELECT} & Passer en mode popup \\
    \hline
    {\bf R+L} & Afficher l'écran d'aide \\
    \hline
    {\bf A} & Sauvegarder dans le slot actuellement sélectionné \\
\hline
\end{supertabular}

\subsubsection{Mode LOAD}
\Image{images/screen_filesystem_load}{1.0}{L'écran FILESYSTEM en mode LOAD}

Le mode LOAD fonctionne sensiblement de la même manière que le mode SAVE: hormis le fait qu'il permet de charger un projet.
Tout comme pour le mode SAVE, une confirmation est nécessaire avant de charger les données.

\Image{images/confirmation_load}{1.0}{Confirmation}

\subsubsection{Commandes}
% Configuration du tableau
\tablehead{\hline \rowcolor{headertab} {\bf Touche(s)} & {\bf Effet} \\ \hline}
\tabletail{\hline \multicolumn{2}{|l|} {\small ...} \\ \hline}
\tablelasttail{\hline}
\begin{supertabular}{|l|p{11cm}|}
\hline
    {\bf SELECT} & Passer en mode popup \\
    \hline
    {\bf R+L} & Afficher l'écran d'aide \\
    \hline
    {\bf A} & Charger le slot actuellement sélectionné \\
\hline
\end{supertabular}

\ColoredAnnotation{Vos morceaux peuvent être enregistrés individuellement sur votre ordinateur !
    Pour cela, utilisez le site web dédié : \url{www.furiousadvancetracker.com}}
