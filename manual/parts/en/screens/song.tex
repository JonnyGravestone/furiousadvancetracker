Cet écran fait office de "séquenceur".
C'est ici que vous allez organiser l'agencement de toutes vos séquences.
Lorsque la lecture de l'ensemble est lancée, \FAT lit toutes les séquences ligne par ligne.
Lorsque un emplacement est vide \FAT essaie de remonter à la première séquence disponible.

\Image{images/screen_song_complete}{1.0}{L'écran SONG se présente comme un gros tableau}

\subsubsection{Commandes}

% Configuration du tableau
\tablehead{\hline \rowcolor{headertab} {\bf Touche(s)} & {\bf Effet} \\ \hline}
\tabletail{\hline \multicolumn{2}{|l|} {\small ...} \\ \hline}
\tablelasttail{\hline}
\begin{supertabular}{|l|p{11cm}|}
\hline
    {\bf SELECT} & Passer en mode popup \\
    \hline
    {\bf R+L} & Afficher l'écran d'aide \\
    \hline
    {\bf START} & Lancer/stopper la lecture depuis le séquenceur \\
    \hline
    {\bf L+HAUT} & Déplacer le curseur tout en haut \\
    \hline
    {\bf L+BAS} & Déplacer le curseur tout en bas \\
    \hline
    {\bf R+HAUT} & Déplacer le curseur d'une page vers le haut \\
    \hline
    {\bf R+BAS} & Déplacer le curseur d'une page vers le bas \\
    \hline
    {\bf A} & Inscrire le dernier numéro de séquence dans la case sous le curseur (par défaut "00") \\
    \hline
    {\bf A+DIRECTION} & Modifier la valeur de la séquence dans la case sous le curseur \\
    \hline
    {\bf L+A} & Trouver une nouvelle séquence disponible puis l'inscrire dans la case sous le curseur \\
    \hline
    %{\bf R+A} & Non utilisé pour le moment \\
    %\hline
    {\bf B} & SI CASE NON VIDE | Couper la séquence \\
    \hline
    {\bf B} & SI CASE VIDE | Coller une séquence préalablement coupée \\
    \hline
    {\bf L+B} & SI CASE VIDE | Coller une séquence préalablement coupée en changeant le numéro \\
    \hline
    {\bf L+B} & SI CASE NON VIDE | Changer le numéro de la séquence en copiant son contenu \\
\hline
\end{supertabular}
