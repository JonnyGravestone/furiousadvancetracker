This screen is the sequencer. Right here, you'll be able to organize your track's sequences order.
When you'll want to play your song, \FAT will read all the sequences line by line.
When an empty spot has been detected, \FAT will try to start from the first in the list.

\Image{images/screen_song_complete}{1.0}{The SONG screen is like a big table.}

\subsubsection{Commands}

% Configuration du tableau
\tablehead{\hline \rowcolor{headertab} {\bf Key(s)} & {\bf Effect} \\ \hline}
\tabletail{\hline \multicolumn{2}{|l|} {\small ...} \\ \hline}
\tablelasttail{\hline}
\begin{supertabular}{|l|p{11cm}|}
\hline
    {\bf SELECT} & Show the switch popup \\
    \hline
    {\bf START} & Start/stop song playback from the sequencer \\
    \hline
    {\bf R+L} & Show help screen \\
    \hline
    {\bf L+UP} & Move the cursor to top \\
    \hline
    {\bf L+DOWN} & Move the cursor to bottom \\
    \hline
    {\bf R+UP} & Move the cursor to one page up \\
    \hline
    {\bf R+DOWN} & Move the cursor to one page down \\
    \hline
    {\bf A} & Write last known sequence's number on the spot designated by cursor (default is "00") \\
    \hline
    {\bf A+DIRECTION} & Modify value \\
    \hline
    {\bf L+A} & Find a new available number and set the value on spot \\
    \hline
    {\bf R+A} & Not used \\
    \hline
    {\bf B} & IF NOT EMPTY SPOT | Cut sequence \\
    \hline
    {\bf B} & IF EMPTY SPOT | Paste sequence \\
    \hline
    {\bf L+B} & IF EMPTY SPOT | Paste sequence but changing number \\
    \hline
    {\bf L+B} & IF NOT EMPTY SPOT | Change number of sequence while copying its content \\
\end{supertabular}
