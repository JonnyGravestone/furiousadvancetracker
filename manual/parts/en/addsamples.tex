\label{sec:addsamples}

As of version 1.0.0, it is possible to add your own samples to \FAT.
This (technical) manipulation is possible on the website (and much simpler, the program on the site takes care of everything)
but you are offered to do it yourself on your own computer if the heart tells you. And that you have nothing better to do.
Let's see how to.
\medskip

\begin{itemize}
  \item{collect your samples in any directory of your choice}
  \item{create a 0infos file and enter the name of your collection: "my-first" for example (8 characters required!) }
  \item{transform your sample files into the right format}
  \item{create the GBFS file that will be embedded in the ROM. Repeat for each kit.}
  \item{in another folder, create the file 00nbkit which counts the number of kits in the final ROM (the number of gbfs therefore)}
  \item{integrate the GBFS files into the "default" ROM}
\end{itemize}\medskip

No worries ! I'll explain everything!
Start by downloading the "default" ROM on the dedicated site.

\paragraph{Format} The first thing to know is the target audio format: {\bf pcm\_s8, 16000Hz, mono}.
This format is very important! If you do not respect it, your samples will not sound properly.
At best you will have noise, at worst you will systematically plant your GBA.

\paragraph{Description file} A collection of samples is described by a file always named {\bf 0infos}.
The content of this text file is very simple: the first 8 letters on the first line are the name of your collection.
The rest is ignored. Make sure to respect these 8 letters (leaving to put '-').

\paragraph{Convert audio files} You are free to use the tool you want.
For my part, I use "ffmpeg" which has the merit to run in command line and is therefore scriptable.
Here is the script I use to transform a complete MP3 collection into the right format:
\Annotation{find . -name "*.mp3" -exec ffmpeg -i {} -ac 1 -ar 16000 -f s8 -acodec pcm\_s8 \{\}.snd \textbackslash;}
Audacity or any other audio processing software is also appropriate.
The important thing is that the format is respected.
For convenience, name your formatted samples with the extension ".snd".

\paragraph{GBFS file} This file is the container that will embed the description file followed by all your .snd file.
To create it, use the utility "gbfs.exe".
You can download this tool on \FAT Github (\href{https://github.com/cbrouillard/furiousadvancetracker/blob/master/gbfs.exe?raw=true}{Github - gbfs.exe})
\Annotation{gbfs.exe monfichier\_sample.gbfs samples/0infos samples/*.snd}
The file "myfile\_sample.gbfs" will be created.
The order of this command is important: the 0infos file MUST be the first.
{\bf This utility works under Linux with Wine}.

\paragraph{00nbkit file} When you have finished creating your GBFS files for each sample kit, a 00nbkit file must be created.
Only one data inside this file: the number of kits that you will integrate into the ROM.
If you have only one kit to integrate:
\Annotation{echo 1 \textgreater  00nbkit}
Once this file is created, it must also be transformed into GBFS.
\Annotation{gbfs.exe 00nbkit.gbfs 00nbkit}

\paragraph{Final ROM integration} Now that your GBFS and 00nbkit.gbfs files are created, just add them to the ROM.
And it's as simple as this command:
\Annotation{cat FAT\_default.gba 00nbkit.gbfs monfichier\_sample.gbfs \textgreater FAT\_v1.0.0\_withsample.gba}
Yes. A simple "cat".
Be careful however, the file 00nbkit.gbfs {\bf MUST} be the first.
(\FAT uses the data written in the file 00nbkit to know the number of kits in advance)
\medskip

After all these manipulations, your ROM is ready: your samples are accessible on SNA and SNB channels with SAMPLE type instruments. Enjoy!
